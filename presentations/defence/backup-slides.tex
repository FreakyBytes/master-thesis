% ------------------------------------------------------------------------------
\section{Backup Slides}
\label{part:backup}

\subsection{KNX}
\begin{frame}[c]
	\centering
	\LARGE Building Automation Systems\\and KNX
\end{frame}

\begin{frame}[c]
	\frametitle{Building Automation Systems}
	\begin{itemize}
		\item Installed in an increasing amount of modern buildings
		\item Serve ever changing requirements for HVAC, energy saving, and monitoring
		\item \textit{\enquote{[...] optimize energy consumption and enable support and maintenance personnel to carry out their jobs more efficiently}} \parencite{Merz2009}
	\end{itemize}
	
	\note{}
\end{frame}

\begin{frame}[c]
	\frametitle{KNX}
	\begin{itemize}
		\item Former European Installation Bus (EIB)
		\item Widely used in Germany and the wider European Market
		\item Typically galvanically isolated 2-wire installation next to mains power line
		\item Addresses separated in 3 Layers: area, line, device
	\end{itemize}
	
	\note{
		Also KNX power line, KNX-IP, KNX-Wireless
	}
\end{frame}

\begin{frame}[c]
	\frametitle{KNX -- Bus Topology}
	\begin{itemize}
		\item Addresses separated in 3 Layers\\area, line, device
		\item Every active device gets assigned a physical address
		\item Devices may listen to group addresses
		\item Different segments are connected by Line Couplers
	\end{itemize}
\end{frame}

\begin{frame}[c]
\frametitle{KNX -- Telegram Types}
	\begin{itemize}
		\item Data Telegram
		\begin{itemize}
			\item Standard Data Telegram ($\leq 16$ Bytes Payload)
			\item Extended Data Telegram ($\leq 255$ Bytes Payload)
		\end{itemize}
		\item Poll Telegram
		\item Acknowledge Telegram
	\end{itemize}
	
	\note{
		
	}
\end{frame}

\begin{frame}[c]
	\frametitle{KNX -- Communication Modes}
	\begin{itemize}
	\item Communication mode determined by TPCI and APCI
	\begin{itemize}
		\item point-to-multipoint, connection-less (multicast)
		\item point-to-domain, connection-less (broadcast)
		\item point-to-all-points, connection-less (system broadcast)
		\item point-to-point, connection-less
		\item point-to-point, connection-oriented
	\end{itemize}
	
	\end{itemize}
	
	\note{
		TPCI -> Transport Layer\\
		APCI -> Application Layer\\
		multicast am häufigsten
	}
\end{frame}

\subsection{Anomaly Detection Algorithms}
\begin{frame}[c]
	\centering
	\LARGE Anomaly Detection Algorithms
\end{frame}

\begin{frame}[c]
	\frametitle{Algorithms -- Unknown Address Detection}
	
	\begin{itemize}
		\item Generate set of occurring source and destination addresses per agent over training period
		\item Compare addresses of captured packages with this set
	\end{itemize}
\end{frame}

\begin{frame}[b]
	\frametitle{Algorithms -- Local Outlier Factor}
	
	\centerImage{110\mm}{../../tex/figures/200-breuning2000-figure3.pdf}{Visualisation of the Local Outlier Factor (LOF) theorem 1 by \textcite{Breunig2000}}{fig:top-threats}
	
	\note{
		\begin{itemize}
			\item Measures the local deviation of density of a given sample compared to its neighbours \parencite{Pedregosa2011}
			\item How isolated is the object in regards of its surrounding? \\ (k-nearest neighbours)
			\item Samples with lower density than their neighbours are consider outliers
			\item Can be trained with unclean data
		\end{itemize}
	}
\end{frame}

\begin{frame}[c]
	\frametitle{Algorithms -- Local Outlier Factor}
	
	\centerImageWithoutCaption{190\mm}{../../tex/figures/200-background-lof.pdf}
\end{frame}

\begin{frame}[c]
	\frametitle{Algorithms -- Support Vector Machine}
	
	\centerImageWithoutCaption{190\mm}{../../tex/figures/200-background-oneclass.pdf}
	
	\note{
		\begin{itemize}
			\item RBF -> Radial Basis Function
			\item OneClass SVM
			\item Full-Gamut of Clean Data required
		\end{itemize}
	}
\end{frame}

\begin{frame}[c]
	\frametitle{Algorithms -- Entropy}
	
	\begin{itemize}
		\item Determine statistical distribution of feature vector dimension in training phase
		\item Calculate Entropy/Information Gain compared to this base distribution
	\end{itemize}
\end{frame}

\subsection{Feature Vector}
\begin{frame}[c]
	\centering
	\LARGE Feature Vector
\end{frame}

\begin{frame}[c]
	\frametitle{Feature Vector Encoding}
	
	\begin{enumerate}
		\item seconds of the week
		\item source address
		\item destination address
		\item priority
		\item hop count
		\item payload length
		\item APCI
	\end{enumerate}
\end{frame}

\begin{frame}[c]
	\frametitle{Feature Vector Encoding}
	
	\begin{itemize}
		\item Encode addresses as bit vector, normalize against probability of appearance in window
		\item Encode priority, hop\_count, payload\_length as single dimension each, normalized against absolute max
		\item Encode APCI via OneHot, but with probability of appearance in window
	\end{itemize}
\end{frame}