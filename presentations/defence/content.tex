% !TeX spellcheck = en_GB
\section{Motivation}
\begin{frame}[c]
	\frametitle{Security in Industrial Control Systems (ICS)}
	\begin{itemize}
		\item 12\% of ICS experienced a breach
		\item 40\% of ICS security professionals can not give clear answers
		\item 69\% consider the threat for ICS severe or high
	\end{itemize}
	\vfill
	{\hfill\footnotesize \textcite{Gregory-Brown2017}}
	
	\note{
		\begin{itemize}
			\item \textbf{Die Sicherheit von Industrie-Steuerungs-Systemen steht auf dem Spiel}.\\
			\item Studie aus 2017 vom SANS Institute 
			\item \textbf{12\%} befragter ICS Sicherheits Exp wurden \textbf{infiltriert}
			\item weitere \textbf{40\% keine Aussage}
			\item \textbf{69\%} schätzen die \textbf{Sicherheitsgefahr auf kritisch}
		\end{itemize}
	}
\end{frame}

\begin{frame}[c]
	\frametitle{Security in Industrial Control Systems (ICS)}
	\centerImage{178\mm}{../../tex/figures/000-gregory-brown2017-threats.png}{Top Threat Vectors in ICS as perceived by security practitioners \parencite{Gregory-Brown2017}}{fig:top-threats}
	
	\note{
		\begin{itemize}
			\item in der gleichen Studie
			\item top threat: \textbf{Devices and \enquote{things} added to the network}
			\item mehr und mehr kleinere Geräte im Netzwerk
			\item nicht (nur) IoT, sondern industrielle/professionelle Anlagen
		\end{itemize}
	}
\end{frame}

\begin{frame}[c]
	\frametitle{Attacks tailored to Embedded Devices increase}
	
	\begin{itemize}
		\item Stuxnet \parencite{Karnouskos2011}
		\item Mirai IoT Botnet \parencite{Kolias2017}
		\item Various exposed and vulnerable ICS devices in IP networks \parencite{Roth2017}
	\end{itemize}
	
	\note{
		\begin{itemize}
			\item Anzahl auf Embedded Devices nimmt zu
			\item Stuxnet -> ICS
			\item Mirai -> IoT Botnet
			\item viele Geräte verwundbar und am Internet, Thomas Roth 34C3
		\end{itemize}
	}
\end{frame}

\begin{frame}[c]
	\frametitle{Building Automation Systems (BAS)}
	
	\begin{itemize}
		\item Despite BAS being a part of ICS
		\item Little work focussing on security BAS
		\item Prior work focussing on Intrusion Detection in BAS by
			\begin{itemize}
				\item \textcite{Yang2006}
				\item \textcite{Celeda2012}
				%\item \textcite{Mundt2012}
				\item \textcite{Pan2014}
			\end{itemize}
	\end{itemize}

	\note{
		\begin{itemize}
			\item Auch wenn BAS Teil von ICS
			\item nur wenig Arbeit spezifisch zu Sicherheit in BAS
			\item Fokus auf Intrusion Detection
		\end{itemize}
		
	}
\end{frame}

\section{Goals and Objective}
\begin{frame}[c]
	\frametitle{Goal}
	
	\begin{itemize}
		\item Increase the security by monitoring BAS
		\item Gather data of possible security incidents
		\item Gain insights of those attacks
	\end{itemize}

	\note{
		\begin{itemize}
			\item (Arbeit im Rahmen von Sindabus)
			\item Übergeordneten Ziele:
			\item Verbesserung der Allg. Sicherheit von BAS
			\item Daten über Infektionen und Angriffe sammeln
			\item Erkenntnisse über Angriffscenarien
		\end{itemize}
	}
\end{frame}

\begin{frame}[c]
	\frametitle{Objective}
	
	\begin{itemize}
		\item Monitor the traffic within BAS
		\item Maintain minimal interference
		%\item Do not rely on existing backbones
		\item Leverage the Netflow concept
		\item Detect threats and intrusions as anomalies
		\item Implement and test a prototype
		\item Determine if an Anomaly-based IDS is a viable concept for BAS
	\end{itemize}
	
	\note{
		\begin{itemize}
			\item BAS traffic überwachen
			\item Minimaler Eingriff \\
				\subitem \hspace{1cm} Keine Zusätzlichen Installationen
			\item Netflow
			\item \emph{Ansatz:} Angriffe als Anomalien erkennen
			\item Prototyp impl + test
			\item Ist Anomalie Erkennung sinnvoll in BAS?
		\end{itemize}
	}
\end{frame}

% ------------------------------------------------------------------------------
\section{The Concept}

\begin{frame}[c]
	\frametitle{KNX as Example}
	
	\centerImageWithoutCaption{160\mm}{../../tex/figures/100-knx-demo-topo.pdf}
	
	\note{
		\begin{itemize}
			\item Schritt zurück
			\item Gebäude-Automatisierungs-System
			\item Komfortfunktionen
			\item ----
			\item KNX als Beispiel
			\item 2-Draht Verkabelung parallel zum Stromkreis
			\item Hier nur logische Struktur wichtig
		\end{itemize}
	}
\end{frame}

\begin{frame}[c]
	\frametitle{KNX as Example}
	
	\centerImageWithoutCaption{160\mm}{../../tex/figures/500-knx-demo-topo-with-agents.pdf}
	
	\note{
		\begin{itemize}
			\item Idee ist dezentral Infos über Aktivitäten sammeln
			\item zu zentraler Stelle weiterleiten
			\item Dazu Agent im Netzwerk
			\item Collector an zentraler Stelle
		\end{itemize}
	}
\end{frame}

\begin{frame}[c]
	\frametitle{The Concept}
	
	\begin{itemize}
		\item<1> Deploy \emph{Agents} in the network collecting traffic information over synchronised time windows
		\item<1-2> Aggregate this information and collect it at a central node
		\item<1-2> Train baseline-models with captured information
		\item<1-2> Use aggregated data to detect deviations from these baseline-models as anomalies
		\item<1-2> Incorporate seasonal sensitivity into those models
		%\item<1-2> \emph{Build an Intrusion Detection System based on the netflow-idea}
			
	\end{itemize}

	\nocite{Jung2018}
	\note{
		\begin{itemize}
			\item Deploy Agents
			\item Infos in Zeit-abschnitten sammeln
			\item Infos zentral sammeln
			\item Infos nutzen um Modelle zu trainieren
			\item Nutze Modelle um neue Beobachtungen zu bewerten
			\item Saisonal sensitivität
			\item ----
			\item Fokus auf Collector
			\item Implementation und Evaluariung des Agents Teil von \textbf{Maximilian Jung}s Arbeit
		\end{itemize}
	}
\end{frame}

\begin{frame}[c]
	\frametitle{Limitation of the Scope}
	
	\begin{itemize}
		\item No classifications of attacks
		\item Focus on anomalies as such
		\item Privacy implications are acknowledged but not further investigated
	\end{itemize}
	
	\note{
		\begin{itemize}
			\item Keine Klassifizierung von Angriffen
			\item Fokus auf Anomalien als solche \\
			\subitem \hspace{1cm} Anomalien im \textbf{Nutzerverhalten} auch erkennen
			\item keine Betrachtung des Datenschutzes \\
			\subitem \hspace{1cm} Implikationen bereits von \textcite{Mundt2012} aufgezeigt
		\end{itemize}
	}
\end{frame}

\begin{frame}[c]
	\frametitle{Monitoring Pipeline}
	
	\centerImageWithoutCaption{180\mm}{../../tex/figures/300-concept-architecture.pdf}
	
	\note{
		\begin{itemize}
			\item orientiert an \textcite{Pan2014}
			\item mehrere Agenten senden agg. Daten
			\item werden in message Queue gesammelt
			\item Collector speicher sofort in InfluxDB
			\item wenn alle Windows für einen Zeitabschnitt da
			\item weiterleiten an Analyser über Message Exchange
			\item ADDR
			\item Entropy (Shannon, Information Gain)
			\item Local Outlier Factor
			\item SVM
			\item speichern Ergebnis direkt in InfluxDB
		\end{itemize}
	}
\end{frame}

\begin{frame}[c]
	\frametitle{Implementation of the Monitoring Pipeline as Prototype}
	
	\begin{itemize}
		\item Prototype implemented as Python modules
		\item Communicating via AMQP and rabbitMQ as message broker
		\item Data is stored in an InfluxDB instance
		\item Anomaly detection based on Python Scikit-learn \parencite{Pedregosa2011}
	\end{itemize}
	
	\note{
		\begin{itemize}
			\item Impl. als Python Modul
			\item Message-passing via AMQP und rabbitMQ
			\item Daten in InfluxDB
			\item Scikit-learn für Datenanalyse Algorithmen
		\end{itemize}
	}
\end{frame}



\begin{frame}[c]
	\frametitle{Selection of Anomaly Detection Algorithms}
	
	\begin{itemize}
		\item Unsupervised Anomaly Detection
		\item Not using signatures or predefined rules
		\item Can be trained without malicious data
		\item Evidence in literature for their performance with regard to IDS \parencite{Lazarevic2003,Toshniwal2014}
	\end{itemize}
	
	\note{
		\begin{itemize}
			\item \textbf{Unsupervised}, aka. ohne weiteren manuellen Einfluss (labelling)
			\item Nicht \textbf{signatur}- oder regel-basierend
			\item Braucht \textbf{keine Daten von zu erkennenden Angriffen}
			\item Wurde bereits in der Literatur als für IDS geeignet erkannt
			\item \textcite{Lazarevic2003,Toshniwal2014}
		\end{itemize}
	}
\end{frame}

% ------------------------------------------------------------------------------
\section{Evaluation}
\begin{frame}[c]
	\centering
	\LARGE Evaluation of the Prototype Implementation
	
	\note{Objective = test performance of proposed IDS concept}
\end{frame}

\begin{frame}[c]
	\frametitle{The Test Setup}
	
	\begin{itemize}
		\item One month of test data from the computer science building
			\begin{itemize}
				\item 2 weeks training data
				\item 1 week validation data
				\item 1 week test data
			\end{itemize}
		\item Test data set is modified with different attack scenarios
	\end{itemize}

	\note{
		\begin{itemize}
			\item 1 Monat Test Daten vom 3. Stock im KZH
			\begin{itemize}
				\item 2w training
				\item 1w validation
				\item 1w test 
			\end{itemize}
			\item Test daten manipuliert
		\end{itemize}
	}
\end{frame}

\begin{frame}[c]
	\frametitle{Simulated Attack Scenarios}
	
	\begin{enumerate}
		\item Unusual traffic
		\item Denial of Service Attack
		\item Network Scan
		\item New Devices
	\end{enumerate}
	
	\note{
		\begin{enumerate}
			\item Unusual traffic
			\item Denial of Service Attack
			\item Network Scan
			\item New Devices
		\end{enumerate}
	}
\end{frame}

\begin{frame}[c]
	\frametitle{Evaluation Criteria}
	
	\begin{enumerate}
		\item General ability to the detect the attack
		\item Differentiation from background noise of the detection results
		\begin{enumerate}[a)]
			\item Average difference in detected outliers 
			\item Average difference in underlying decision metric
		\end{enumerate}
		\item Response time 
		\item Persistence of detection
	\end{enumerate}
	
	\note{
		\begin{enumerate}
			\item Wurde Attacke erkannt
			\item Abgrenzung vom Hintergrund rauschen
			\begin{enumerate}[a)]
				\item in der Anzahl erkannter Anomalien
				\item in der zugrundeliegenden Metrik
			\end{enumerate}
			\item Erkennungszeit
			\item Persistenz
		\end{enumerate}
	}
\end{frame}

\section{Detection Results of the DoS Attack}
\begin{frame}[c]
	\centering
	\LARGE Detection Results of the DoS Attack
	
	\note{}
\end{frame}

\begin{frame}[c]
	\frametitle{Telegrams with Unknown Addresses During the DoS Attack}
	
	\centerImageWithoutCaption{230\mm}{../../tex/figures/700-results/dos-tc.png}
\end{frame}

\begin{frame}[c]
	\frametitle{Outliers of SVM and LOF classifier during the DoS Attack}
	
	\centerImageWithoutCaption{230\mm}{../../tex/figures/700-results/dos-num-outliers.png}
\end{frame}

\begin{frame}[c]
	\frametitle{Spread of LOF during the DoS Attack}
	
	\centerImageWithoutCaption{230\mm}{../../tex/figures/700-results/dos-lof-spread.png}
\end{frame}

\begin{frame}[c]
	\frametitle{Spread of SVM Distance during the DoS Attack}
	
	\centerImageWithoutCaption{230\mm}{../../tex/figures/700-results/dos-svm-spread.png}
\end{frame}

\begin{frame}[c]
	\frametitle{Average Entropy during the DoS Attack}
	
	\centerImageWithoutCaption{230\mm}{../../tex/figures/700-results/dos-entropy.png}
\end{frame}

\begin{frame}[c]
	\frametitle{Detection Results of the DoS Attack}
	
	\begin{table}[H]
		\aboverulesep=0ex
		\belowrulesep=0ex
		\renewcommand{\arraystretch}{1.2}
		\newcolumntype{Y}{>{\centering\arraybackslash}X}
		
		\small
		\centering
		\begin{tabularx}{0.95\textwidth}{|l|Y|Y|Y|Y|}
			\toprule
			& LOF & SVM & Address & Entropy \\\midrule
			1. Attack detected & yes & yes & yes & no \\\midrule
			2.a) Avg. difference in outliers  & $0.549$ & $0.997$ & $254$ & n/a \\\midrule
			2.b) Avg. difference in metric & $-1.730$ & $-39.46$ & n/a & n/a \\\midrule
			3. Response time & 0s & 0s & 0s & n/a \\\midrule
			4. Persistence & $100$\% & $100$\% & $100$\% & n/a \\\bottomrule
		\end{tabularx}
		%\caption[Detection results of the DoS attack]{Detection results of the \gls{dos} attack. Difference in averages is calculated against the validation dataset.}
		%\label{tab:results:dos}
	\end{table}
\end{frame}

\begin{frame}[c]
	\frametitle{Summary of Detection Results}
	
	\begin{table}[h]
		\aboverulesep=0ex
		\belowrulesep=0ex
		\renewcommand{\arraystretch}{1.2}
		\newcolumntype{Y}{>{\centering\arraybackslash}X}
		
		\centering
		\begin{tabularx}{0.95\textwidth}{|l|Y|Y|Y|Y|}
			\toprule
			& LOF & SVM & Address & Entropy \\\midrule
			Unusual Traffic & no & no & no & no \\\midrule
			DoS Attack & yes & yes & yes & no \\\midrule
			Network Scan & no & yes & yes & no \\\midrule
			New Devices & no & no & yes & no \\\midrule
		\end{tabularx}
		%\caption[Summary of detection results]{Summary of detection results.}
		%\label{tab:results:conclusion}
	\end{table}
\end{frame}

% ------------------------------------------------------------------------------
\section{Results and Conclusion}
\begin{frame}[c]
	\centering
	\LARGE Results and Conclusion
	
	\note{}
\end{frame}

\begin{frame}[c]
	\frametitle{Results}
	
	\begin{itemize}
		\item \emph{Unusual Traffic} modification was not detected at all
		\item \emph{New Devices} were only identified address analyser
		\item \emph{DoS} and \emph{Network Scan} clearly identified
	\end{itemize}

	\note{
		\begin{itemize}
			\item \emph{Unusual traffic} war nicht ungewöhnlich genug
			\item \emph{New Devices} gleiches Problem, \\
				\subitem \hspace{1cm} da traffic von selber Line, aber anderer Tag
			\item sesonal sensitivity nur eingeschränkt gezeigt
			\item ----
			\item \emph{DoS} und \emph{Network Scan} relativ leicht erkennbar, da 
			\item ----
			\item In aktueller Form gut, um Manipulationen im Bus zu erkennen
			\item Weniger gut als Alarmanlage \\
				\subitem \hspace{1cm} -> \textbf{WAS AUCH NIE ZWECK WAR}
		\end{itemize}
	}
\end{frame}

\begin{frame}[c]
	\frametitle{Results}
	
	\begin{itemize}
		\item SVM seems to be the best fit for IDS
		\item LOF requires more parameter tuning
		\item Entropy failed to detect anything
		\item Address analyser performed good for its simplicity
	\end{itemize}
	
	\note{
		\begin{itemize}
			\item SVM best fit -> genaue Detection, kleine Models, schnelle Erkennung
			\item LOF -> sensitiv zu Anzahl an Neighbours und MinPts\\
				\subitem \hspace{1cm} riesige Modelle
			\item Entropy -> fehlerhafte Impl. oder ungeeignet\\
				\subitem \hspace{1cm} nicht nach \textcite{Toshniwal2014}, da eher Clustering
			\item Address -> einfaches aber äußerst wirksames Mittel
		\end{itemize}
	}
\end{frame}

\begin{frame}[c]
	\frametitle{Possible Error Sources}
	
	\begin{itemize}
		\item Poor choice of parameters
		\item Implementation problems
		\item Mistakes by encoding information into the feature vector
		\item Unsatisfactory test data
	\end{itemize}

	\note{
		\begin{itemize}
			\item Uninformierte Wahl von ML Parametern
			\item Impl. Probs
			\item Feature Vector
			\item Nicht ausreichende Test Daten (nicht variant genug, zu wenig, ...)
		\end{itemize}
	}
\end{frame}

\begin{frame}[c]
	\frametitle{Conclusion}
	
	\begin{itemize}
		\item Proposed and showcased a modular and extensible monitor pipeline
		\item Showed the feasibility of using in-band monitoring messages
		\item Concept of an anomaly-based IDS in BAS was validated
		\item Multiple approaches to anomaly detection were benchmarked in different test scenarios
	\end{itemize}

	\note{
		\begin{itemize}
			\item Vorstellung modularer und erweiterbarer Pipeline
			\item Nutzbarkeit von in-band monitoring messages
			\item IDS in BAS
			\item Mehrer Anomaly Detect. Algos
		\end{itemize}
	}
\end{frame}



\begin{comment}
\begin{frame}[c]
	\frametitle{Motivation}
	%\centering
	\textit{\enquote{4 out of 10 ICS security practitioners lack visibility or sufficient supporting intelligence into their ICS networks}}
	
	\hspace{60mm} --- \textcite{Gregory-Brown2017}
\end{frame}
\end{comment}
