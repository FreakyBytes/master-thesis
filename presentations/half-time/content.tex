% !TeX spellcheck = en_GB

% ------------------------------------------------------------------------------
\section{Motivation}
\begin{frame}[c]
	\frametitle{Motivation}
	\begin{itemize}
		\item Ever increasing inter-linkage of Building-, Automotive-, and Industrial-Networks
		\item concerns are internal (accidental) threats \parencite{Gregory-Brown2017}
			\begin{itemize}
				\item many devices insecure by design
				\item wide spread legacy technology
				\item connectivity amplifies threat
			\end{itemize}
		\item \textit{\enquote{[...] threat from nearly every vector identified by ICS security practitioners
			can be reduced by detailed monitoring of ICS network traffic [...]}} \parencite{Gregory-Brown2017}
	\end{itemize}
	
	\note{}
\end{frame}

\begin{frame}[c]
	\frametitle{Motivation}
	%\centering
	\textit{\enquote{4 out of 10 ICS security practitioners lack visibility or sufficient supporting intelligence into their ICS networks}}
	
	\hspace{60mm} --- \textcite{Gregory-Brown2017}
\end{frame}

\begin{frame}[c]
	\frametitle{Motivation}
	\begin{itemize}
		\item Situation worse for Building Automation Systems
		\item Leverage existing installations and infrastructure
		\item Provide cheap reliable method to increase security in Building Automation Systems
	\end{itemize}
\end{frame}

% ------------------------------------------------------------------------------
\part{Background}
\label{part:back}

\section{Building Automation Systems}
\begin{frame}[c]
	\centering
	\LARGE Building Automation Systems
\end{frame}

\begin{frame}[c]
	\frametitle{Building Automation Systems}
	\begin{itemize}
		\item Installed in an increasing amount of modern buildings
		\item Serve ever changing requirements for HVAC, energy saving, and monitoring
		\item \textit{\enquote{[...] optimize energy consumption and enable support and maintenance personnel to carry out their jobs more efficiently}} \parencite{Merz2009}
	\end{itemize}
	
	\note{}
\end{frame}

\subsection{KNX}
\begin{frame}[c]
	\frametitle{KNX}
	\begin{itemize}
		\item Former European Installation Bus (EIB)
		\item Widely used in Germany and the wider European Market
		\item Typically galvanically isolated 2-wire installation next to the mains power line
	\end{itemize}

	\note{
		Also KNX power line, KNX-IP, KNX-Wireless
	}
\end{frame}

\begin{frame}[c]
	\frametitle{KNX -- Bus Topology}
	\begin{itemize}
		\item Addresses separated in 3 Layers\\area, line, device
		\item Every active device gets assigned a physical address
		\item Devices may listen to group addresses
		\item Different segments are connected by Line Couplers
	\end{itemize}
\end{frame}

\begin{frame}[c]
	\frametitle{KNX -- Bus Topology}
	
	\centerImage{155\mm}{./figures/knx-demo-topo.pdf}{KNX Demo Topology}{fig:knx-demo-topo}
	\nocite{Sokollik2017}
	
	\note{
		Area -> Line -> Device
		Physical 
	}
\end{frame}

\begin{frame}[c]
	\frametitle{KNX -- Telegram Types}
	\begin{itemize}
		\item Data Telegram
			\begin{itemize}
				\item Standard Data Telegram ($\leq 16$ Bytes Payload)
				\item Extended Data Telegram ($\leq 255$ Bytes Payload)
			\end{itemize}
		\item Poll Telegram
		\item Acknowledge Telegram
	\end{itemize}
	
	\note{
		
	}
\end{frame}

\begin{frame}[c]
	\frametitle{KNX -- Communication Modes}
	\begin{itemize}
		\item Communication mode determined by TPCI and APCI
			\begin{itemize}
				\item point-to-multipoint, connection-less (multicast)
				\item point-to-domain, connection-less (broadcast)
				\item point-to-all-points, connection-less (system broadcast)
				\item point-to-point, connection-less
				\item point-to-point, connection-oriented
			\end{itemize}
			
	\end{itemize}
	
	\note{
		TPCI -> Transport Layer
		APCI -> Application Layer
		multicast am häufigsten
	}
\end{frame}


% ------------------------------------------------------------------------------
\part{Concept}
\label{part:concept}

\begin{frame}[c]
	\centering
	\LARGE Concept
\end{frame}

\begin{frame}[c]
\frametitle{Conceptual Idea to detect attacks in BAS}
\begin{itemize}
	\item \textcolor{lightgray}{Deploy agents in the network}
	\item Gather data in a collector to obtain world-view
	\item Compare this data to a base-line model to detect \textit{anomalies}
	\item aka. \textit{adapt netflow for BAS}
\end{itemize}
\end{frame}

\section{Netflow}
\begin{frame}[c]
	\frametitle{What is Netflow?}
	\begin{itemize}
		\item Passive network monitoring
		\item Packets are aggregated into flows \parencite{Hofstede2014}
		\item Flow $\coloneqq$ set of packets passing an observation point during a certain time interval with common properties \parencite{Claise2013}
		%\item Enables to monitor high throughput networks
		\item in-band
	\end{itemize}
\end{frame}

\section{Research Questions}
\begin{frame}[c]
	\frametitle{Research Questions}
	\begin{enumerate}
		\item \textit{\enquote{[...] anomaly detection methods seem especially applicable to SCADA system security which are characterized by routine and repetitious activities.}} \parencite{Yang2006}\\
			Is it also a good fit for BAS? Does it make sense?
		\item How can anomaly detection identify different attack vectors in BAS:\\
			{\small new devices, (high) traffic load, network problems, configuration changes}
		\item In which way does anomaly detection need to account for different seasons?
	\end{enumerate}
\end{frame}

\begin{frame}[c]
\frametitle{Research Questions}
	\begin{enumerate}
		\setcounter{enumi}{3}
		\item Does the additional in-band traffic influences normal operations?
		\item Does the additional in-band traffic produces new attack surfaces?
		\item Which anomaly detection/evaluation method îs the best for BAS?
		\item Which data reduction is feasible and in which part of the system does it makes sense? (in the data collection agent)
	\end{enumerate}
\end{frame}

\section{Known Limitations}
\begin{frame}[c]
	\frametitle{Known Conceptual Limitations}
	\begin{itemize}
		\item No classification of attacks (kind or severity)
		\item Abnormal behaviour is also an anomaly
		\item Requires training data
		\item eavesdropping is not detectable
		%\item Possibly using a sledgehammer to crack a hazel nut
	\end{itemize}
\end{frame}

% ------------------------------------------------------------------------------
\part{Architecture}
\label{part:architecture}

\section{Architecture}
\begin{frame}[c]
	\centering
	\LARGE Architecture
\end{frame}

\begin{frame}[c]
	\frametitle{Architecture}
	
	\centerImageWithoutCaption{200\mm}{./figures/architecture.pdf}
	
	\note{
	}
\end{frame}

\begin{frame}[c]
	\frametitle{Architecture}
	
	\centerImageWithoutCaption{200\mm}{./figures/unknown-addr.png}
	
	\note{
	}
\end{frame}

\subsection{Algorithms}
\begin{frame}[c]
	\centering
	\LARGE Algorithms
\end{frame}

\begin{frame}[c]
	\frametitle{Algorithms -- Unknown Address Detection}
	
	\begin{itemize}
		\item Generate set of occurring source and destination addresses per agent over training period
		\item Compare addresses of captured packages with this set
	\end{itemize}
\end{frame}

\begin{frame}[c]
	\frametitle{Algorithms -- Local Outlier Factor}
	
	\begin{itemize}
		\item Measures the local deviation of density of a given sample compared to its neighbours \parencite{Pedregosa2011}
		\item How isolated is the object in regards of its surrounding? \\ (k-nearest neighbours)
		\item Samples with lower density than their neighbours are consider outliers
		\item Can be trained with unclean data
	\end{itemize}
\end{frame}

\begin{frame}[c]
	\frametitle{Algorithms -- Entropy}
	
	\begin{itemize}
		\item Determine statistical distribution of feature vector dimension in training phase
		\item Calculate Entropy/Information Gain compared to this base distribution
	\end{itemize}
\end{frame}

\subsection{Feature Vector}
\begin{frame}[c]
	\frametitle{Feature Vector}
	
	\begin{itemize}
		\item Analysed Variance and Standard Derivation
		\item source, destination, priority, hop\_count, APCI, payload\_length
	\end{itemize}
	\centerImage{100\mm}{./figures/variance-feature-vector.png}{Variance of KNX telegram fields}{fig:variance}
\end{frame}

\begin{frame}[c]
	\frametitle{Feature Vector Encoding}
	
	\begin{itemize}
		\item Encode addresses as bit vector, normalize against probability of appearance in window
		\item Encode priority, hop\_count, payload\_length as single dimension each, normalized against absolute max
		\item Encode APCI via OneHot, but with probability of appearance in window
	\end{itemize}
\end{frame}

% ------------------------------------------------------------------------------
\part{Outlook}
\label{part:outlook}

\section{Outlook}
\begin{frame}[c]
	\centering
	\LARGE Outlook
\end{frame}

\begin{frame}[c]
	\frametitle{Outlook -- What's left to do}
	\begin{itemize}
		\item Finish and test implementation of LOF and Entropy analyser
		\item Tune machine learning parameter
		\item Write up
	\end{itemize}
\end{frame}
