% !TeX spellcheck = en_GB

\begin{itemize}
	\item experiment
	\item test data captured from a floor section of the computer science building
	\item enriched with malicious packets to keep consistent
	
	\item (focusses only an purpose based) attack classes (cf. \parencite{Uma2013})
		\subitem \gls{dos}
			\subsubitem Short circuit -> blackout on entire line
			\subsubitem flooding of \code{A\_Restart} telegrams
			\subsubitem flooding nonsense
		\subitem replay
			\subsubitem repeating a time window
			\subsubitem sniff a tag and repeat it compressed??? \alert{whatever this means?}
		\subitem telegram manipulation
		\subitem reconfiguration of devices (Access Attack)
		\subitem reconfigure line couplers/make them useless (Access Attack)
		\subitem spoofing 
		\subitem Reconnaissance Attack
			\subsubitem network mapping
			\subsubitem do inverse action
		\subitem 
	
	\item aim is to show if attacks can be identified by anomaly detection on flow data
		\subitem under the assumption, that attacks noticeable alter the characteristic and behaviour of \gls{knx} traffic
		\subitem cf. \parencite{Mukherjee1994,Yang2006,Pan2014}
	\item demonstrate a message-passing architecture to perform online analytics on \gls{knx} flow-data
	\item benchmark different algorithms against each other
\end{itemize}

\begin{comment}
Angriffe:

DoS
	Kurzschluss im Bus -> DoS auf gesamtem Segment
	A_Restart-Pakete -> DoS gegen einzelne Teilnehmer
Replay-Angriffe
	Zeit mitschneiden -> wiedergeben
	Tag mitschneiden, komprimiert wiedergeben
Manipulation von Paketen (Payload tauschen)
Konfiguration manipulieren
Überwindung von Linienkopplern
Address-Spoofing
	falsche Adresse in Liniensegment
	mit existierender Adresse senden
Netzanalyse mit knxMap (https://github.com/takeshixx/knxmap)
Mitlesen und sofort gegenteilige Aktion auslösen
High-Level-Angriffe:
	nur best. Aktionen zulassen
	Provokation/Sabotage von menschl. Verhalten
Social-Engineering -> Einschleusen von Geräten
\end{comment}