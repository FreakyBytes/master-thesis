% !TeX spellcheck = en_GB

\begin{comment}
\begin{itemize}
	\item experiment
	\item test data captured from a floor section of the computer science building
	\item enriched with malicious packets to keep consistent
	
	\item (focusses only an purpose based) attack classes (cf. \parencite{Uma2013})
		\subitem \gls{dos}
			\subsubitem Short circuit -> blackout on entire line
			\subsubitem flooding of \code{A\_Restart} telegrams
			\subsubitem flooding nonsense
		\subitem replay
			\subsubitem repeating a time window
			\subsubitem sniff a tag and repeat it compressed??? \alert{whatever this means?}
			\subsubitem do inverse action
		\subitem manipulation/reconfiguration
			\subsubitem telegram manipulation
			\subsubitem reconfiguration of devices (Access Attack)
			\subsubitem reconfigure line couplers/make them useless (Access Attack)
		\subitem spoofing 
		\subitem Reconnaissance Attack
			\subsubitem network mapping
			
		\subitem 
	
	\item aim is to show if attacks can be identified by anomaly detection on flow data
		\subitem under the assumption, that attacks noticeable alter the characteristic and behaviour of \gls{knx} traffic
		\subitem cf. \parencite{Mukherjee1994,Yang2006,Pan2014}
	\item demonstrate a message-passing architecture to perform online analytics on \gls{knx} flow-data
	\item benchmark different algorithms against each other
\end{itemize}
\end{comment}

The objective of this thesis is to show, that anomaly based \glsfirst{ids} can provide additional security benefits for \glsfirst{bas} in the same they does for \gls{scada} networks. (cf. Section~\ref{sec:intro})
Given the assumption, that every malicious activity or attack within a network induces noticeable different characteristics and behaviour, these deviations should be also identifiable in aggregated flow-data of this network. \parencite[cf.][]{Mukherjee1994,Yang2006,Pan2014}
For this I designed a message-passing architecture to test different established algorithms for anomaly detection using flow-data from \gls{bas} networks.

The test will be conducted as an experiment with data captured in a floor segment of the computer science building over the course of one month. \alert{check, if I can get a second month for verifying the training?} It is assumed, that this data set does not contain any malicious activities.
Consequently, this data will be used to train multiple models using different algorithms.
Then a comparable data set is used to ensure that the models are not over-fitted and do not raise alarms during normal operations.
Further, the same validation data set is modified with malicious activities, which might be cause by a range of plausible attacks scenarios.
This, in turn, is down to ensure that the models and algorithms are not under-fitted and are suitable to detect anomalies, which might be induced by malicious activities.
However, the experiment will not focus on determining, if the system can distinguish between malicious abnormal activities and legit abnormal activities, which might be caused by a rapid change of user behaviour. In this thesis both are considered worth reporting.

Mostly since rapidly changed user behaviour is difficult to simulate in our test case, the actual test and benchmarking will be done using a set of plausible attacks. \hint{Is this even the right place to describe the attacks?}
These attacks include \emph{\glsfirst{dos} attacks}, which can be induced by multiple action: First of all shorting the line circuit would effectively causing a communication blackout on the entire line. Also flooding \code{A\_Restart} telegrams to all devices, would render the devices useless, since they would be stuck in a restart loop. As last scenario of \gls{dos} attacks, flooding the line with non-sense telegrams on high priority (cf. Section~\ref{sec:background:bas:knx:proto}) would cause at least heavily delayed, if not entirely blocked, normal communication on the line, since no time window would be left open for normal priority telegrams.

Another plausible attack scenario would be replay attacks, where 3 flavours might be imaginable.
In the first case an attacker would captures traffic for a determined time span and then replays it on the network, possibly to mimic normal behaviour while breaking into the building.
Secondly, an attacker could capture an specific event and replay it on will, which could the command to open an security door for instance.
As a third option, the traffic could be monitored and for every action an reverse action could be invoked, effectively keeping the \gls{bas} in one state. The simplest example would be, to turn off the lights every time somebody tries to turn them on.

The next category of attacks can be classified as manipulation or reconfigure attacks. This includes attacks which might modify telegrams while they are send, to either render them invalid and therefore preventing communication. Or it would be conceivable to reconfigure devices, i.e. to trigger different actions, report false measurements etc. A specialisation of this attack focusses on line couplers and breaking the network segmentation by disabling any routing functions.

Finally, the last category describes reconnaissance attacks, which focus on unauthorised detection and mapping of the network. Here only active sweeping approaches are considered, since passive eavesdropping can not be detected on higher protocol level due to the bus character of the network. (cf. Section~\ref{sec:background:bas:knx:topo})

\hint{check plausibility of this}
\todo{write more about evaluating the results?}
For each category of attacks, the different anomaly detection algorithms are benchmarked with regards to their detection rate.





\begin{comment}
Angriffe:

DoS
	Kurzschluss im Bus -> DoS auf gesamtem Segment
	A_Restart-Pakete -> DoS gegen einzelne Teilnehmer
Replay-Angriffe
	Zeit mitschneiden -> wiedergeben
	Tag mitschneiden, komprimiert wiedergeben
Manipulation von Paketen (Payload tauschen)
Konfiguration manipulieren
Überwindung von Linienkopplern
Address-Spoofing
	falsche Adresse in Liniensegment
	mit existierender Adresse senden
Netzanalyse mit knxMap (https://github.com/takeshixx/knxmap)
Mitlesen und sofort gegenteilige Aktion auslösen
High-Level-Angriffe:
	nur best. Aktionen zulassen
	Provokation/Sabotage von menschl. Verhalten
Social-Engineering -> Einschleusen von Geräten
\end{comment}