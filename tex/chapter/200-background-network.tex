% !TeX spellcheck = en_US

\section{General Network Monitoring?}

\section{Anomaly Detection?}

\section{Netflow}
\begin{itemize}
	\item active
		\begin{itemize}
			\item inject additional traffic to measure things \parencite{Hofstede2014}
			\item e.g. ping, traceroute
		\end{itemize}
	\item passive
		\begin{itemize}
			\item "observe existing traffic as it passes by" \parencite{Hofstede2014}
			\item e.g. packet capture "This method generally provides most insight into the network traffic, as complete packets can be captured and further analyzed. However, in high-speed networks with line rates of up to 100 Gbps, packet capture requires expensive hardware and substantial infrastructure for storage and analysis." \parencite{Hofstede2014}
			\item "Another passive network monitoring approach that is more scalable for use in high-speed networks is flow export, in which packets are aggregated into flows and exported for storage and analysis." \parencite{Hofstede2014}
			\item  "a set of IP packets passing an observation point in the network during a certain time interval, such that all packets belonging to a particular flow have a set of common properties." \parencite{Claise2013}
			
		\end{itemize}
	\item 
\end{itemize}

\section{Intrusion Detection Systems (IDS)}
	\subsection{Signature Intrusion Detection Systems (SIDS)}
	\subsection{Anomaly-based Intrusion Detection Systems (ABIDS)}
	\subsection{State-based Intrusion Detection Systems\hint{??}}

\section{Anomaly and Novelty Detection}

	\subsection{General Approaches}
	\begin{itemize}
		\item cf. \textcite{Hodge2004}
		\item Type 1: "Determine the outliers with no prior knowledge of the data" \parencite{Hodge2004}
		\subitem "unsupervised clustering" \parencite{Hodge2004}
		\subitem "assumes that errors or faults are separated from the ‘normal’ data and will thus appear as outliers" \parencite{Hodge2004}
		\subitem " predominantly retrospective and is analogous to a batch-processing system" \parencite{Hodge2004} - Requires a bunch of data to begin with
		\subitem " requires that all data be available before processing" \parencite{Hodge2004}
		\subitem "data is static" \parencite{Hodge2004}
		\subitem "two sub-techniques" \parencite{Hodge2004}
		\subitem -> "outlier diagnostic approach highlights the potential outlying points" \parencite{Hodge2004} and "prune the outliers and fit their system model to the remaining data until no more outliers are detected" \parencite{Hodge2004}
		\subitem -> "accommodation that incorporates the outliers into the distribution model generated and employs a robust classification method" \parencite{Hodge2004} which "can withstand outliers in the data and generally induce a boundary of normality around the majority of the data" \parencite{Hodge2004}
		\subitem "Non-robust methods are best suited when there are only a few outliers in the data set" \parencite{Hodge2004}
		\subitem "a robust method must be used if there are a large number of outliers to prevent this distortion" \parencite{Hodge2004}
		\item Type 2: "Model both normality and abnormality." \parencite{Hodge2004}
		\subitem "supervised classification" \parencite{Hodge2004}
		\subitem "requires pre-labelled data" \parencite{Hodge2004}
		\subitem "The entire area outside the normal class represents the outlier class" \parencite{Hodge2004}
		\subitem "best suited to static data" \parencite{Hodge2004} "classification needs to be rebuilt from first principles if the data distribution shifts" \parencite{Hodge2004}
		\subitem "can be used for on-line classification" \parencite{Hodge2004} learning while new samples are classified
		\subitem "require a good spread of both normal and abnormal data" \parencite{Hodge2004}
		\subitem "classification is limited to a ‘known’ distribution" \parencite{Hodge2004}
		\subitem "cannot always handle outliers from unexpected regions" \parencite{Hodge2004}
		\item Type 3: "Model only normality [...]" \parencite{Hodge2004}
		\subitem "novelty detection" \parencite{Hodge2004} " semi-supervised recognition or detection task" \parencite{Hodge2004}
		\subitem "needs pre-classified data but only learns data marked normal" \parencite{Hodge2004}
		\subitem "suitable for static or dynamic data" \parencite{Hodge2004}
		\subitem "can learn the model incrementally as new data arrives" \parencite{Hodge2004}
		\subitem "aims to define a boundary of normality" \parencite{Hodge2004}
		\subitem "requires the full gamut of normality to be available for training to permit generalisation" \parencite{Hodge2004} but "requires no abnormal data" \parencite{Hodge2004}
		\subitem this is good, because "Abnormal data is often difficult to obtain or expensive in many fault detection domains [...]" \parencite{Hodge2004}
		\subitem "as long as the new fraud lies outside the boundary of normality then the system will be correctly detect the fraud" \parencite{Hodge2004}
		
	\end{itemize}
	
	\subsection{Anomaly Detection Methods}
	
	\subsubsection{Local Outlier Factor}
	\begin{itemize}
		\item seems to be a good fit
		\item works on unclean training data
		\item good selection of vectors and distance functions is required
		\item 11\% of the first 500 telegrams in \verb|eiblog.txt| are considered outliers
		\item density based
	\end{itemize}
	
	\subsubsection{On Class SVM}
	\begin{itemize}
		\item aka. novelty detection
		\item all training data is considered good
	\end{itemize}
	
	\subsubsection{Isolation Forest}
	\begin{itemize}
		\item ...
	\end{itemize}
	
	\subsubsection{Elliptical Envelope}
	\begin{itemize}
		\item tries to fit data to estimated \emph{shape}
		\item does not seem to be a good match, since it required to know the distribution of the vector fields
	\end{itemize}

	\subsection{Statistical BlaBla}
	\subsection{Entropy based}
	\subsection{Bayesian}
	\subsection{Pattern Matching}
	\subsection{Autoassociative Kernel Regression (AAKR)}
		used by \textcite{Yang2006}

\section{Time-based Anomaly Detection}

\section{Prior Work and Existing Solutions}
\hint{as own chapter?}
\begin{itemize}
	\item \parencite{Yang2006}
	\item \parencite{Celeda2012}
	\item \textcite{Pan2014} BACnet
		\begin{itemize}
			\item Anonmaly detection for BACnet fire alarm systems
			\item taps BACnet traffic via IP
			\item rule based learning (RIPPER)
			\item protection against common BACnet attack vectors
				\subitem who-is/who-has network probing
				\subitem write-property (take control over device)
				\subitem InitializedRoutingTable
				\subitem reinitialize devices
				\subitem application layer DoS
				\subitem flooding of network
		\end{itemize}
\end{itemize}