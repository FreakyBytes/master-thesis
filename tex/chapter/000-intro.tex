% !TeX spellcheck = en_US

The security of the industry and office buildings is on stake.
Modern functional builds are often equipped with \glspl{bas} to provide additional comfort functions and cost saving features.
These advantages are enabled by a network of interconnected devices, which measure, control, and regulate various aspects like \gls{hvac}.
Unfortunately, security aspects are seldom considered in such installations. \parencite{Brandstetter2017}
Established protocols often lack strong encryption or these features are turned of in deployment.

According to a survey by \textcite{Gregory-Brown2017}, \enquote{Devices and \enquote{things} (that cannot protect themselves) added to network} are leading among the top three threat vectors of concern.
With the increasing popularity of Ethernet networks, more and more \gls{bas} are connected to them via gateways. This exposes additional attack surfaces.
It is to note, that \textcite{Gregory-Brown2017} surveyed to security of \gls{ics} of which \gls{bas} are only a small portion.
Even more worrying appears that over 52\% of the questioned security professionals could not give founded and clear answers, if they experienced a security breach in their \glspl{ics}.

\section{Motivation}
\begin{itemize}
	
	\item "[...] the overall concerns are the internal threat (accidental) and the increasing presence of connected devices, many insecure by design, in and around ICS environments." \parencite[p.~9]{Gregory-Brown2017}
	\item "The threat from nearly every vector identified by ICS security practitioners can be reduced by detailed monitoring of ICS network traffic6 in a manner that provides visibility into both process anomalies and security anomalies on the control network, in some cases establishing control points limiting access to different zones of your network" \parencite[p.~10]{Gregory-Brown2017}
	\item "4 out of 10 ICS security practitioners lack visibility or sufficient supporting intelligence into their ICS networks" \parencite[p.~13]{Gregory-Brown2017}
\end{itemize}

\section{Scope of this work}

\section{Negative scope of this work}
\begin{itemize}
	\item only \knx
	\item no real impl of the agent, focus only on collector
	\item no optimization for high throughput
	\item no infrastructure for deploy'n'forget (no automatic learning etc.)
		\subitem this is research, installation requires manual steps
	\item no classification of intrusion?
	\item well aware of privacy implications, but not part of this work
\end{itemize}

\section{Research Questions}

\begin{enumerate}
	\item \enquote{[...] anomaly detection methods seem especially applicable to SCADA system security which are characterized by routine and repetitious activities.} \parencite{Yang2006} Is it also a good fit for BAS? Does it make sense?
	\item How can anomaly detection identify different attack vectors in BAS:
		\subitem new devices
		\subitem (high) traffic load
		\subitem network problems
		\subitem configuration changes
		\subitem reconnaissance (network scan) 
	\item In which way does anomaly detection need to account for different seasons? And how high is the precision improvement compared to a model that is not season-sensible?
	\item Does the additional in-band traffic influences normal operations?
	\item Does the additional in-band traffic produces new attack surfaces?
	\item Which anomaly detection/evaluation method îs the best for BAS?
	\item Which data reduction is feasible and in which part of the system does it makes sense? (in the data collection agent)
\end{enumerate}