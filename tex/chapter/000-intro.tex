% !TeX spellcheck = en_US
\section{Motivation}
\begin{itemize}
	\item "[...] the overall concerns are the internal threat (accidental) and the increasing presence of connected devices, many insecure by design, in and around ICS environments." \parencite[p.~9]{Gregory-Brown2017}
	\item "The threat from nearly every vector identified by ICS security practitioners can be reduced by detailed monitoring of ICS network traffic6 in a manner that provides visibility into both process anomalies and security anomalies on the control network, in some cases establishing control points limiting access to different zones of your network" \parencite[p.~10]{Gregory-Brown2017}
	\item "4 out of 10 ICS security practitioners lack visibility or sufficient supporting intelligence into their ICS networks" \parencite[p.~13]{Gregory-Brown2017}
\end{itemize}

\section{Scope of this work}

\section{Research Questions}

\begin{enumerate}
	\item \enquote{[...] anomaly detection methods seem especially applicable to SCADA system security which are characterized by routine and repetitious activities.} \parencite{Yang2006} Is it also a good fit for BAS? Does it make sense?
	\item How can anomaly detection identify different attack vectors in BAS:
		\subitem new devices
		\subitem (high) traffic load
		\subitem network problems
		\subitem configuration changes
	\item In which way does anomaly detection need to account for different seasons? And how high is the precision improvement compared to a model that is not season-sensible?
	\item Does the additional in-band traffic influences normal operations?
	\item Does the additional in-band traffic produces new attack surfaces?
	\item Which anomaly detection/evaluation method îs the best for BAS?
	\item Which data reduction is feasible and in which part of the system does it makes sense? (in the data collection agent)
\end{enumerate}