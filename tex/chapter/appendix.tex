% !TeX spellcheck = en_GB
% ------------------------------------------------------------------------------
% code highlighting
% some nice styling for code listings
\definecolor{mygreen}{rgb}{0,0.6,0}
\definecolor{mygray}{rgb}{0.5,0.5,0.5}
\definecolor{mymauve}{rgb}{0.58,0,0.82}
\definecolor{lightgray}{rgb}{0.97,0.97,0.97}
%
\lstset{ %
	backgroundcolor=\color{lightgray},   % choose the background color; you must add \usepackage{color} or \usepackage{xcolor}
	basicstyle=\footnotesize,        % the size of the fonts that are used for the code
	breakatwhitespace=false,         % sets if automatic breaks should only happen at whitespace
	breaklines=true,                 % sets automatic line breaking
	captionpos=b,                    % sets the caption-position to bottom
	commentstyle=\color{mygreen},    % comment style
	deletekeywords={...},            % if you want to delete keywords from the given language
	escapeinside={\%*}{*)},          % if you want to add LaTeX within your code
	extendedchars=true,              % lets you use non-ASCII characters; for 8-bits encodings only, does not work with UTF-8
	frame=single,	                   % adds a frame around the code
	framerule=0pt,
	keepspaces=true,                 % keeps spaces in text, useful for keeping indentation of code (possibly needs columns=flexible)
	%keywordstyle=\color{blue},       % keyword style
	%keywordstyle={},
	language=Octave,                 % the language of the code
	otherkeywords={*,...},           % if you want to add more keywords to the set
	numbers=left,                    % where to put the line-numbers; possible values are (none, left, right)
	numbersep=5pt,                   % how far the line-numbers are from the code
	numberstyle=\tiny\color{mygray}, % the style that is used for the line-numbers
	rulecolor=\color{black},         % if not set, the frame-color may be changed on line-breaks within not-black text (e.g. comments (green here))
	showspaces=false,                % show spaces everywhere adding particular underscores; it overrides 'showstringspaces'
	showstringspaces=false,          % underline spaces within strings only
	showtabs=false,                  % show tabs within strings adding particular underscores
	stepnumber=2,                    % the step between two line-numbers. If it's 1, each line will be numbered
	%stringstyle=\color{mymauve},     % string literal style
	stringstyle=\textit,
	tabsize=2,	                   % sets default tabsize to 2 spaces
	title=\lstname                   % show the filename of files included with \lstinputlisting; also try caption instead of title
}
%
% ------------------------------------------------------------------------------

\chapter{Additional Visualisations of Anomaly-Detection Metrics}
\chaptermark{Additional Visualisations}
\label{app:metrics}
\newpage

\section{Metrics During the Training Phase}
\label{app:metrics:training}

\begin{figure}[H]
	\newcommand{\figwith}{0.49\textwidth}
	\newcommand{\figprefix}{train}
	\centering
	
	\begin{subfigure}[b]{\figwith}
		\includegraphics[width=\textwidth]{figures/700-results/\figprefix-tc.png}
		\caption{Telegrams (green) and telegrams with unknown addresses (yellow) over time}
		\label{fig:results:\figprefix:tc}
	\end{subfigure}
	\hfil
	\begin{subfigure}[b]{\figwith}
		\includegraphics[width=\textwidth]{figures/700-results/\figprefix-unknown-addr.png}
		\caption{Amount of observed unknown source (green) and destination (yellow) addresses}
		\label{fig:results:\figprefix:addr}
	\end{subfigure}
	\\[1.5mm]
	\begin{subfigure}[b]{\figwith}
		\includegraphics[width=\textwidth]{figures/700-results/\figprefix-num-outliers.png}
		\caption{Number of detected outliers via SVM (red) and LOF (blue).}
		\label{fig:results:\figprefix:outlier}
	\end{subfigure}
	\hfil
	\begin{subfigure}[b]{\figwith}
		\includegraphics[width=\textwidth]{figures/700-results/\figprefix-entropy.png}
		\caption{Calculated Entropy over time, infinity is encoded as $100 000$}
		\label{fig:results:\figprefix:entropy}
	\end{subfigure}
	\\[1.5mm]
	\begin{subfigure}[b]{\figwith}
		\includegraphics[width=\textwidth]{figures/700-results/\figprefix-lof-spread.png}
		\caption{Spread and average of the Local Outlier Factor, smaller numbers indicate outlier}
		\label{fig:results:\figprefix:lof}
	\end{subfigure}
	\hfil
	\begin{subfigure}[b]{\figwith}
		\includegraphics[width=\textwidth]{figures/700-results/\figprefix-svm-spread.png}
		\caption{Spread and average of the distance to decision surface, negative distances are outlier}
		\label{fig:results:\figprefix:svm}
	\end{subfigure}
	
	\caption[Detection Results During the Training Phase]{Detection Results During the Training Phase from 2017-01-21 00:00 to 2017-02-04 23:59 with a minimal time resolution of 1 hour.}
	\label{fig:results:\figprefix}
	
\end{figure}

\section{Metrics During the Validation Phase}
\label{app:metrics:validation}

\begin{figure}[H]
	\newcommand{\figwith}{0.49\textwidth}
	\newcommand{\figprefix}{validation}
	\centering
	
	\begin{subfigure}[b]{\figwith}
		\includegraphics[width=\textwidth]{figures/700-results/\figprefix-tc.png}
		\caption{Telegrams (green) and telegrams with unknown addresses (yellow) over time}
		\label{fig:results:\figprefix:tc}
	\end{subfigure}
	\hfil
	\begin{subfigure}[b]{\figwith}
		\includegraphics[width=\textwidth]{figures/700-results/\figprefix-unknown-addr.png}
		\caption{Amount of observed unknown source (green) and destination (yellow) addresses}
		\label{fig:results:\figprefix:addr}
	\end{subfigure}
	\\[1.5mm]
	\begin{subfigure}[b]{\figwith}
		\includegraphics[width=\textwidth]{figures/700-results/\figprefix-num-outliers.png}
		\caption{Number of detected outliers via SVM (red) and LOF (blue).}
		\label{fig:results:\figprefix:outlier}
	\end{subfigure}
	\hfil
	\begin{subfigure}[b]{\figwith}
		\includegraphics[width=\textwidth]{figures/700-results/\figprefix-entropy.png}
		\caption{Calculated Entropy over time, infinity is encoded as $100 000$}
		\label{fig:results:\figprefix:entropy}
	\end{subfigure}
	\\[1.5mm]
	\begin{subfigure}[b]{\figwith}
		\includegraphics[width=\textwidth]{figures/700-results/\figprefix-lof-spread.png}
		\caption{Spread and average of the Local Outlier Factor, smaller numbers indicate outlier}
		\label{fig:results:\figprefix:lof}
	\end{subfigure}
	\hfil
	\begin{subfigure}[b]{\figwith}
		\includegraphics[width=\textwidth]{figures/700-results/\figprefix-svm-spread.png}
		\caption{Spread and average of the distance to decision surface, negative distances are outlier}
		\label{fig:results:\figprefix:svm}
	\end{subfigure}
	
	\caption[Detection Results During the Validation Phase]{Detection Results During the Validation Phase from 2017-02-04 00:00 to 2017-02-11 23:59 with a minimal time resolution of 1 hour.}
	\label{fig:results:\figprefix}
	
\end{figure}

\chapter{Steps to Reproduce the Experiment Results}
\chaptermark{Reproduction Steps}
\label{app:reproduce}

Before attempting to build the sources, please ensure that following software packages are installed and working:
\code{Git}, \code{Docker 17.05.0-ce}, \code{docker-compose 1.19.0}, \code{Python 3.6.4+}, \code{Virtualenv 13.1.2}, and \code{pip 9.0.1}.

All described steps were developed and tested on
\code{Linux 4.14.0-3-amd64 \#1 SMP Debian 4.14.17-1 (2018-02-14) x86\_64 GNU/Linux}.\\
Alternative platforms might need adjustments to the build scripts or steps.

% set some new settings for listings
\lstset{
	language=Bash,
	breaklines=true,
	breakatwhitespace=true,
	numbers=none,
	frame=single,
	framerule=0pt,
	belowskip=-24pt,
	aboveskip=5pt,
	basicstyle=\footnotesize\ttfamily,
	stringstyle={},
	keywordstyle={},
}

\section{Build and Install the Prototype}
\label{app:reproduce:build}

\begin{enumerate}
	\item Clone the git repository or alternatively use the source repository on the attached disk in Appendix~\ref{app:disk}
		\begin{enumerate}
			\item Clone the main repository
\begin{lstlisting}
git clone https://github.com/FreakyBytes/master-thesis.git
\end{lstlisting}
			\item Navigate into the newly created directory
\begin{lstlisting}
cd master-thesis
\end{lstlisting}			
			\item Clone all submodules
\begin{lstlisting}
git submodule init
git submodule update
\end{lstlisting}
		\end{enumerate}
	\item Create a temporary Python 3 virtual environment, the directory should be changed for more permanent installations
\begin{lstlisting}
virtualenv -p $(which python3) /tmp/bob-venv
\end{lstlisting}
	\item Activate the virtual environment. \textbf{This step must be repeated whenever a new terminal session is started.}
\begin{lstlisting}
source /tmp/bob-venv/bin/activate
\end{lstlisting}
	\item Install the \gls{baos} parser library
\begin{lstlisting}
pip install --process-dependency-links --editable src/baos-knx-parser
\end{lstlisting}
	\item Install the \code{BOb} command line tool
\begin{lstlisting}
pip install --process-dependency-links --editable src/bas-observe
\end{lstlisting}
	\item Check if everything was installed correctly
\begin{lstlisting}
bob --help
\end{lstlisting}
\end{enumerate}

\section{Setup Required Services}
\label{app:reproduce:services}

\begin{enumerate}
	\item Start a new terminal session, not using the virtual environment
	\item Ensure the docker daemon is up and running
\begin{lstlisting}
docker info
\end{lstlisting}
	\item Navigate in from the main repository to the \code{bas-observe} submodule's \code{docker} directory
\begin{lstlisting}
cd src/bas-observer/docker
\end{lstlisting}
	\item Start the services in the background using \code{docker-compose}
\begin{lstlisting}
docker-compose up -d
\end{lstlisting}
	\item Check the status of the container
\begin{lstlisting}
docker-compose ps
\end{lstlisting}
	\item Optionally, monitor the log output from \code{InfluxDB}, \code{RabbitMQ}, and \code{Grafana}
\begin{lstlisting}
docker-compose logs -f --tail=200
\end{lstlisting}
	\item Log into \code{Grafana} by navigating to \url{http://localhost:3000/} using a web browser\\
		Username: \code{admin}\\
		Password: \code{testpw}
	\item Import the pre-made dashboards, by navigating to \emph{create} shown as a plus on the left side, and then \emph{Import}
	\item Upload \code{src/bas-observer/docker/grafana-bob-dashboard-avg.json}
	\item Repeat the upload for \code{src/bas-observer/docker/grafana-bob-dashboard-sum.json}
\end{enumerate}

\section{Train the Models}
\label{app:reproduce:train}

\begin{enumerate}
	\item Ensure the services are set up and are running, according to Appendix~\ref{app:reproduce:services}
	\item Activate the virtual environment if necessary, this needs to be repeated for every new terminal session
\begin{lstlisting}
source /tmp/bob-venv/bin/activate
\end{lstlisting}
	\item Start the Collector
\begin{lstlisting}
bob -l INFO --project test collector -a agent0
\end{lstlisting}
	\item Start the the Simulator Agent to import the training dataset
\begin{lstlisting}
bob -l INFO --project test simulate --dump-format new --agent agent0 0 0 data/dataset_training.csv
\end{lstlisting}
	\item To accelerate the import process additional Collectors can be started, without relaying function
\begin{lstlisting}
bob -l INFO --project test collector -a agent0 --no-relay
\end{lstlisting}
	\item Wait until every every window is processed by the Collector\\ (not just until the Simulator Agent terminates)
	\item Train the four different models
		\begin{enumerate}
			\item \gls{lof} Analyser
\begin{lstlisting}
bob -l DEBUG --project test train lof --start "2012-02-27T00:00:00" --end "2012-02-27T01:00:00" -m models/lof_model.json
\end{lstlisting}
			\item \gls{svm} Analyser
\begin{lstlisting}
bob -l DEBUG --project test train svm --start "2012-02-27T00:00:00" --end "2012-02-27T01:00:00" -m models/svm_model.json
\end{lstlisting}
			\item Entropy Analyser
\begin{lstlisting}
bob -l DEBUG --project test train entropy --start "2012-02-27T00:00:00" --end "2012-02-27T01:00:00" -m models/entropy_model.json
\end{lstlisting}
			\item Address Analyser
\begin{lstlisting}
bob -l DEBUG --project test train addr --start "2012-02-27T00:00:00" --end "2012-02-27T01:00:00" -m models/addr_model.json
\end{lstlisting}
		\end{enumerate}
\end{enumerate}

\section{Analyse the Validation and Test Datasets}
\label{app:reproduce:run}

\begin{enumerate}
	\item Ensure the services are set up and are running, according to Appendix~\ref{app:reproduce:services}
	\item Activate the virtual environment if necessary, this needs to be repeated for every new terminal session
\begin{lstlisting}
source /tmp/bob-venv/bin/activate
\end{lstlisting}
	\item Start the Collector
\begin{lstlisting}
bob -l INFO --project test collector -a agent0
\end{lstlisting}
	\item Additional Collector can be started to speed up the processing
\begin{lstlisting}
bob -l INFO --project test collector -a agent0 --no-relay
\end{lstlisting}
	\item Start all Analyser modules
		\begin{enumerate}
			\item \gls{lof} Analyser
\begin{lstlisting}
bob -l INFO --project test analyse lof -m models/lof_model.json
\end{lstlisting}
			\item \gls{svm} Analyser
\begin{lstlisting}
bob -l INFO --project test analyse svm -m models/svm_model.json
\end{lstlisting}
			\item Entropy Analyser
\begin{lstlisting}
bob -l INFO --project test analyse entropy -m models/entropy_model.json
\end{lstlisting}
			\item Address Analyser
\begin{lstlisting}
bob -l INFO --project test analyse addr -m models/addr_model.json
\end{lstlisting}
		\end{enumerate}
	\item Import the validation dataset
\begin{lstlisting}
bob -l INFO --project test simulate --dump-format new --agent agent0 0 0 data/dataset_validation.csv
\end{lstlisting}
	\item Import the test dataset
\begin{lstlisting}
bob -l INFO --project test simulate --dump-format new --agent agent0 0 0 data/dataset_test.csv
\end{lstlisting}
\end{enumerate}

\begin{comment}
\begin{lstlisting}
\end{lstlisting}
\end{comment}

\chapter{Disc with Software and Data}
\label{app:disk}