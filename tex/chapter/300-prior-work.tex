% !TeX spellcheck = en_GB

%\section{Prior Work and Existing Solutions}
%\label{sec:background:network:priorwork}

\begin{itemize}
	\item \parencite{Yang2006}
	\item \parencite{Celeda2012}
	\item \textcite{Pan2014} BACnet
	\begin{itemize}
		\item Anonmaly detection for BACnet fire alarm systems
		\item taps BACnet traffic via IP
		\item rule based learning (RIPPER)
		\item protection against common BACnet attack vectors
		\subitem who-is/who-has network probing
		\subitem write-property (take control over device)
		\subitem InitializedRoutingTable
		\subitem reinitialize devices
		\subitem application layer DoS
		\subitem flooding of network
	\end{itemize}
	
	\item \textcite{Eskin2002} algos for unsupervised anomaly detection in IDS
	\item \textcite{Leung2005} Unsupervised anomaly detection in IDS, proposed new algo \emph{pfMAFIA}
\end{itemize}

Despite being widely used and applied in \gls{ip} networks \parencite[cf.][pp.~201~ff.]{Northcutt2005}, \gls{ids} seem rather seldom utilized to monitor \glspl{bas}.
However, there are a few examples in literature where the principles of \glspl{ids} are applied to \gls{scada} and \gls{ics}.

One example of deploying anomaly in \gls{scada} networks is presented by \textcite{Yang2006}, with the base assumption that a security violation changes the system behaviour and that this change can be detected. They argue, that the repeatable and predictable nature of traffic in \gls{scada} networks make them a application field for anomaly detection based \glspl{ids}, opposed to rule based.
Further, they focus on inside threads where the attacker already has gained access and some knowledge about the targeted systems and possible vulnerabilities.
The proposed method is to use a \gls{aakr} model for anomaly detection, to predict \emph{corrected} versions of their selected traffic features. Then the binary hypothesis technique \gls{sprt} was applied, to determine if the observed sequence was likely to be generated by normal behaviour or not.
\textcite{Yang2006} show, that using \gls{aakr} with \gls{sprt} anomalies can identify malicious traffic in \gls{scada} networks, however they also note that it is crucial to monitor a large variety of system measurements and it is therefore an important task to carefully determine which variables are of value and which are not.

The threads in \gls{scada} networks, acknowledged by \textcite{Yang2006}, are also recognised by \textcite{Celeda2012}. They note, that not only \gls{scada} networks but also \gls{bas} provide more and more attack surfaces and exploiting those can have huge implications for companies and the society as a whole.
Further, security considerations in these networks are still highly underestimated, compared to \gls{ip} networks.
Consequently, \textcite{Celeda2012} argue, that not only the passive security (cf. Section~\ref{sec:background:bas:knx:security}) needs to be increased, but intrusions and malicious behaviour has to be detected. For this they investigate possible applications and the advantages of flow-based monitoring (cf. Section~\ref{sec:background:network:netflow}) in \gls{bas}, specifically in the example of \gls{bac}.
The flow analysis part focuses on \gls{bac} over \gls{ip} and is analysed with a simple volumetric approach (simply measuring the throughput) and by calculating the entropy (cf. Section~\ref{sec:background:network:novelty:entropy}) of the flow data.
As a result \textcite{Celeda2012} were able to detect and identify several attacks and a bot net in \gls{bac} installations, based on flow data.

Further investigation in \gls{bac} security is published by \textcite{Pan2014}.


