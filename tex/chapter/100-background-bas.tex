% !TeX spellcheck = en_US

\section{Introduction into Building Automation}
	\begin{itemize}
		\item Increasing amount of complexity and requirement of comfort in private and commercial buildings \parencite{Merz2009}

		\item benefits regarding saving and managing energy \parencite{Merz2009}
		\item Ever changing requirements: "in commercial buildings flexibility is high on the agenda -- offices buildings, for example, should be designed in such way that they can be easily adapted to meet any change in use or requirements" \parencite{Merz2009}
		\item "In modern buildings there are variety of automation systems for heating, ventilating and air conditioning" \parencite{Merz2009}
		\item "control systems optimize energy consumption and enable support and maintenance personnel to carry out their jobs more efficiently" \parencite{Merz2009}
		
	\end{itemize}

\section{\knx}
	
	\begin{itemize}
		\item \knx or Konnex
		\item "formerly known as European Installation Bus (EIB)" \parencite{Merz2009}
		\item "designed to be used in electrical installations for implementing automated functions and processes in buildings" \parencite{Merz2009}
		\item "Different transmission media can be used for the bus: twisted-pair cable (KNX.TP), power line (KNX.PL), radio frequency (KNX.RF) and fiber-optic cable" \parencite{Merz2009}
		\item focus on KNX.TP
			\subitem 2 variants
			\subitem TP-0: 2400 Baud unbalanced, derived from BatiBUS \parencite{CENELEC2004}
			\subitem TP-1: 9600 Baud balanced, derived from EIB (relevant here) \parencite{CENELEC2004}
		\item KNX.PL and KNX.RF for integration into older buildings \parencite{Merz2009}
		\item Standardized as DIN EN 50090 written by \textcite{CENELEC2004}
		\item "Large sections was approved for inclusion into ISO/IEC 14543" \parencite{Merz2009}
		\item "World wide only open standard for home and building control" \parencite{Merz2009}
		\item Benefits of \knx include
			\subitem more devices due to different manufactures
			\subitem large variety of devices (sensors, actuators, control/regulation, operation and monitoring)
	\end{itemize}
	
	\subsection{Bus Topology of KNX Networks}
	\begin{itemize}
		\item "Like a conventional electrical installation, a KNX installation needs to have a power line network to provide the loads with electricity. But it also has a communication network – the KNX installation bus" \parencite{Merz2009}
		\item "Both networks are galvanically isolated" \parencite{Merz2009}
		\item 2 separate cable installations as a result
		\item \knx is designed to reshape the structure of conventional building installations, in form of a tree topology \parencite{Merz2009}
		\item "Nodes are assigned to a line" \parencite{Merz2009}
		\item "Several lines are connected to a main line and form an area" \parencite{Merz2009}
		\item "Several areas are connected with each other via the back bone line" \parencite{Merz2009}
		\item Each node is assigned an address, defining area, line, and device number which forms the logical topology
			\subitem \hint{include bit field table for phy addrs}
		\item It is advised, that the physical topology reassembles the logical \parencite{Sokollik2017}
		\item Even though, the physical \knx bus can build in various forms (star, tree, ring, and mixed forms) \parencite{Sokollik2017}
		\item Most significant subdivision of the physical topology in \knx.TP-1 is a line segment \parencite{Sokollik2017}
			\subitem a segment defines how many \knx devices can be connected on a physical line \parencite{Sokollik2017}
			\subitem segment is a part of the bus system, which connects \knx devices electrical continuous with each other. \parencite{Sokollik2017}
		\item 2 types of \knx.TP-1 devices \parencite{Sokollik2017}
			\subitem TP1-64 and TP1-256 cf. Table \ref{tab:background:bas:knx:topo:tpsegments}
			\subitem mainly differ in the maximum amount of devices, which can be connected in one segment
			\subitem problem since market is very fragmented
			\subitem products are often not clearly with regards to TP1-64 and TP1-256
			\subitem better of assuming TP1-64 for installations, since one TP1-64 device in a segment forces the whole segment to be downgraded to the TP1-64 standard
		\item therefore max a full logical line must be build in four segments with 3 line couplers \parencite{Sokollik2017}
			\subitem it is to note that is not to advice since a \knx telegram can only take 6 hops
			\subitem e.g. can be transmitted by max 6 couplers
		\item area is created by coupling up to 16 lines \parencite{Sokollik2017}
			\subitem 1 main line and 15 sublines
			\subitem lines are coupled via line repeater to the main line
			\subitem line couplers have own physical address (device address 0) \parencite{Sokollik2017}
			\subitem line couplers galvanically separate main line and sub line, are capable of routing \parencite{Sokollik2017}
			
		
		
	\end{itemize}

	\begin{table}
		\begin{tabular}{l r r }
			 & \textbf{TP1-64} & \textbf{TP1-256} \\\hline
			 maximum devices per segment & 64 & 256 \\
			 maximum distance between 2 devices & 700m & 700m \\
			 maximum length of a physical line & 1000m & 1000m \\
		\end{tabular}
		\caption[Segment properties for \knx.TP1-64 and \knx.TP1-256]{Segment properties for \knx.TP1-64 and \knx.TP1-256. cf. \textcite{Sokollik2017}}
		\label{tab:background:bas:knx:topo:tpsegments}
	\end{table}
	
	\subsection{KNX Protocol}
		\begin{itemize}
			\item 3 types of packets
				\subitem standard
				\subitem extended
				\subitem poll
		\end{itemize}
	
	\subsection{Security in KNX}
	\subsection{Examples and Applications of KNX Networks}

\section{\lonworks}

\section{\bacnet}