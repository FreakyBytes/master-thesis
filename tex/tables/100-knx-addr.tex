% KNX Address byte-field

\begin{table}[b]
	\aboverulesep=0ex
	\belowrulesep=0ex
	\renewcommand{\arraystretch}{1.2}
	
	\centering
	\begin{tabular}{|l|c|c|c|c|c|c|c|c|c|c|c|c|c|c|c|c|}
		\toprule
		Byte & \multicolumn{8}{c|}{1} & \multicolumn{8}{c|}{0} \\\midrule
		Bit & 7 & 6 & 5 & 4 & 3 & 2 & 1 & 0 & 7 & 6 & 5 & 4 & 3 & 2 & 1 & 0\\\midrule
		Physical Address & \multicolumn{4}{c|}{area} & \multicolumn{4}{c|}{line} & \multicolumn{8}{c|}{device}\\\midrule
		Group Address & \multicolumn{5}{c|}{main} & \multicolumn{3}{c|}{middle} & \multicolumn{8}{c|}{sub group}\\
		\bottomrule
	\end{tabular}
	\caption[Bit field for \knx addresses]{Bit field for \knx addresses. Only the most common used 3-level group address is shown here. cf.~\textcite{Merz2009,Sokollik2017} }
	\label{tab:background:bas:knx:topo:addr}
\end{table}