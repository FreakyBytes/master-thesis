% !TeX spellcheck = en_US
\documentclass[12pt, a4paper, titlepage, oneside]{book}
\usepackage[a4paper, margin=3cm]{geometry}

%
\usepackage[utf8]{inputenc}			% utf-8, bitches
\usepackage[T1]{fontenc}			% das Trennen der Umlaute
\usepackage[english]{babel}			% word wrapping

\usepackage{parskip}				% makes paragraph spacing nicer
\usepackage{setspace}				% more spacing control
%\setstretch{1.25}

\usepackage[breaklinks,hidelinks]{hyperref}		% links

%\usepackage{lmodern}
%\usepackage{textcomp}

\usepackage{xspace}					% intelligent spacing after words, usefull for shortcut coammnds
\usepackage{graphicx}				% better graphics support
\usepackage{color}					% enables support for using different colors
\usepackage{csquotes}				% show quotes

\usepackage{comment}
\usepackage{caption}
\usepackage{float}					% better positioning for graphics, tables, etc.
%\usepackage{tabu}					% better tables -> unmaintained?
\usepackage{booktabs}				% improvements to LaTeX tables
\usepackage{tabularx}				% better tables
%\usepackage{multirow}				% for cells spanning multiple rows
%\usepackage{multicol}				% seems to be necessary for multirow?
\usepackage{calc}					% enables calculations in heights and stuff
\usepackage{wrapfig}				% figures on the side
\usepackage{pdfpages}				% include whole pdfs as pages

\usepackage[title,toc,page]{appendix}		% appendix environment
\usepackage{listings}				% source code listings
%usepackage{longtable}				% multipage tables
%usepackage{array}					% enables the user of m{} and b{} in tables
%usepackage{multirow}				% allow multiple rows

%usepackage{pdflscape}				% produces landscape pages (landscape env)

\usepackage{amssymb}				% math
\usepackage{amsmath}				% more math
\usepackage{textgreek}				% allows for greek chars
\usepackage[style=iso]{datetime2}	% date format, that makes sense to everybody

\usepackage[acronym,toc]{glossaries}	% Glossary and Acronym support

\usepackage[backend=bibtex, style=authoryear,
	citestyle=authoryear, sorting=ynt,
	sortcites=true, block=none,	indexing=true,
	citereset=none, isbn=true,
	url=true, doi=true, natbib=false,
	hyperref=true]{biblatex}					% even better citation support
\addbibresource{bas-sec.bib}
\addbibresource{standards.bib}

%
\usepackage{ccicons}				% icons for the creative commons licenses

%
\usepackage{hyperxmp}				% pdf meta information
\hypersetup{
	pdfauthor={Martin Peters},
	pdfauthortitle={Master Student},
	pdfcreator={},
	pdflang={en},
	pdfkeywords={Building Automation, KNX, netflow},
	pdftype={Text},
	pdfcopyright={Copyright (C) 2017/2018, Martin Peters.},
	pdflicenseurl={}
}

% ------------------------------------------------------------------------------
% Shortcuts
% 
\definecolor{pink}{rgb}{1,0,1}
\definecolor{red}{rgb}{1,0,0}

\newcommand{\hint}[1]{{\color{pink}#1}}
%\newcommand{\hint}[1]{}

\newcommand{\alert}[1]{{\color{red}#1}}
%\newcommand{\alert}[1]{}

% Shortcut file

% ------------------------------------------------------------------------------
% Shortcuts
% 
\definecolor{pink}{rgb}{1,0,1}
\definecolor{red}{rgb}{1,0,0}

\newcommand{\hint}[1]{{\color{pink}#1}}
%\newcommand{\hint}[1]{}

\newcommand{\alert}[1]{{\color{red}#1}}
%\newcommand{\alert}[1]{}

% ------------------------------------------------------------------------------
% Context shortcuts
% 

\newcommand{\hvac}{HVAC\xspace}
\newcommand{\knx}{KNX\xspace}
\newcommand{\lonworks}{LonWorks\xspace}
\newcommand{\bacnet}{BACnet\xspace}

% ------------------------------------------------------------------------------
% Glossary
% 
\makeglossaries
% ------------------------------------------------------------------------------
% Glossary
% 

% BAS stuff
\newacronym[plural=BAS,firstplural=Building Automation Sytems (BAS)]{bas}{BAS}{Building Automation System}
\newacronym{hvac}{HVAC}{heating, ventilation, and air conditioning}

\newglossaryentry{knx}{name={KNX},description={KNX or Konnex}}
\newglossaryentry{lon}{name={LonWorks},description={LonWorks}}
\newglossaryentry{bac}{name={BACnet},description={BACnet}}

% KNX terminology
\newacronym{apci}{APCI}{Application Layer Protocol Control Information}
\newacronym{tpci}{TPCI}{Transport Layer Protocol Control Information}

\newglossaryentry{telegram}{name={Telegram},description={\gls{knx} Packet}}
\newglossaryentry{hops}{name={hop count},description={\todo{hop count}}}

% ML stuff
\newglossaryentry{vect}{name={vector space},description={}}

\newacronym{svm}{SVM}{support vector maschine}
\newacronym{lof}{LOF}{Local Outlier Factor}

% other stuff
\newglossaryentry{py}{name={Python},description={Interpreted Programming Language}}
\newglossaryentry{influxdb}{name={InfluxDB},description={NoSQL time series database}}
\newglossaryentry{idbmeasurement}{name={measurement},plural={measurements},description={An InfluxDB measurement is comparable to a table in relational databases. It contains the timestamp and tags as indecies and fields as data}} %\url{https://docs.influxdata.com/influxdb/v1.4/concepts/key_concepts/#measurement}
\newglossaryentry{grafana}{name={Grafana},description={Open Source time series visualisation and monitoring dashboard}}
\newglossaryentry{rabbitmq}{name={RabbitMQ},description={Open Source message passing server, implementing \gls{amqp}. \\\url{http://}}}
\newglossaryentry{lib-click}{name={\code{click}},description={Python package for creating beautiful command line interfaces in a composable way with as little code as necessary. \\\url{http://click.pocoo.org/6/}}}
\newglossaryentry{boas}{name={KNX BAOS},description={Protocol communicate with \gls{knx} modules developed by Weinzierl Engineering. \\\url{https://www.weinzierl.de/index.php/en/}}}

\newacronym{amqp}{AMQP}{Advanced Message Queuing Protocol}
\newacronym{gil}{GIL}{global interpreter lock}
%\newglossaryentry{gil}{name={GIL},description={Global Interpreter Lock. A mechanism used by the default CPython implementation to ensure that Python bytecode is only executed by one thread at a time}}

% stats stuff
\newacronym{pmf}{PMF}{propability mass function}


% ------------------------------------------------------------------------------
% Titel
% 
\newcommand{\thedate}{\today}					% TODO change before handing in
\newcommand{\theauthor}{Martin Peters}
\newcommand{\thetitle}{Analysis of Distributed in-band Monitoring Messages for Field Bus Networks in Building Automation Systems}

\title{\thetitle\\[24pt]
	\small Chair for Information and Communication Services\\[-3pt]
	\small University of Rostock}
\author{\theauthor}
\date{\thedate}
%
% ------------------------------------------------------------------------------

\begin{document}
	% begin with no page numbering
	\pagenumbering{gobble}
	% title
	%\maketitle
	% TitelSeite
\begin{titlepage}

	\begin{center}
		
		\hrule
		\vspace{0.5cm}
		\LARGE {\bfseries \thetitle}
		\vspace{0.5cm}
		\hrule
		
		\vspace{0.8cm}
		\Large {Master Thesis}\\
		\vspace{1.0cm}
		\large {University of Rostock\\
		Faculty of Computer Science and Electrical Engineering\\
		Institute of Computer Science\\
		Department of Information- and Communication-Systems}
		
		%\vspace{0.8cm}
		\vfill
		\includegraphics[width=7cm]{style/UNI-Logo.pdf}
		%\vspace{0.8cm}
		
	\end{center}
	
	\vfill
	\large
	\begin{tabular}{lcl}
		Submitted by: &&  \theauthor \\
		Matriculation Number: && 212206972\\
		Born at: && 1993-09-30 in Schwerin\\
		First Reviewer: && Prof. Dr. rer. nat. Clemens H. Cap\\
		Second Reviewer: && \\
		Supervisor: && Dr.-Ing. Thomas Mundt\\
		Submission Date: && \alert{dummy Wert}
	\end{tabular}
	\normalsize
	
	
\end{titlepage}
	%
	% compiling info
	~ \vfill
	\begin{flushright}
		\tiny \noindent
		{\normalsize \href{http://creativecommons.org/licenses/by-sa/4.0/}{\ccbysa}} \\*[0.5em]
		This work is licensed under a \href{http://creativecommons.org/licenses/by-sa/4.0/}{Creative Commons\\%*[-1em]
		Attribution-ShareAlike 4.0 International License}. \\
		The license of the supplementary material especially source code\\%*[-1em]
		may differ. Please refer to the respective LICENSE files.\\
		\LaTeX ~compiled at \DTMnow
	\end{flushright}
	\pagebreak
	%
	% --------------------------------------------------------------------------
	%
	% begin with roman page numbering
	\clearpage
	\setcounter{page}{1}
	\pagenumbering{roman}
	% the toc
	\tableofcontents
	% the lof
	\listoffigures
	% the lot
	\listoftables
	
	% --------------------------------------------------------------------------
	% begin of actual document
	%
	% begin new page numbering
	\newpage
	\pagenumbering{arabic}
	\setstretch{1.25}
	
	\chapter{Introduction}
	\label{sec:intro}
	% !TeX spellcheck = en_US

The security of the  is on stake.
...

\section{Motivation}
\begin{itemize}
	\item "[...] the overall concerns are the internal threat (accidental) and the increasing presence of connected devices, many insecure by design, in and around ICS environments." \parencite[p.~9]{Gregory-Brown2017}
	\item "The threat from nearly every vector identified by ICS security practitioners can be reduced by detailed monitoring of ICS network traffic6 in a manner that provides visibility into both process anomalies and security anomalies on the control network, in some cases establishing control points limiting access to different zones of your network" \parencite[p.~10]{Gregory-Brown2017}
	\item "4 out of 10 ICS security practitioners lack visibility or sufficient supporting intelligence into their ICS networks" \parencite[p.~13]{Gregory-Brown2017}
\end{itemize}

\section{Scope of this work}

\section{Negative scope of this work}
\begin{itemize}
	\item only \knx
	\item no real impl of the agent, focus only on collector
	\item no optimization for high throughput
	\item no infrastructure for deploy'n'forget (no automatic learning etc.)
		\subitem this is research, installation requires manual steps
	\item no classification of intrusion?
	\item well aware of privacy implications, but not part of this work
\end{itemize}

\section{Research Questions}

\begin{enumerate}
	\item \enquote{[...] anomaly detection methods seem especially applicable to SCADA system security which are characterized by routine and repetitious activities.} \parencite{Yang2006} Is it also a good fit for BAS? Does it make sense?
	\item How can anomaly detection identify different attack vectors in BAS:
		\subitem new devices
		\subitem (high) traffic load
		\subitem network problems
		\subitem configuration changes
		\subitem reconnaissance (network scan) 
	\item In which way does anomaly detection need to account for different seasons? And how high is the precision improvement compared to a model that is not season-sensible?
	\item Does the additional in-band traffic influences normal operations?
	\item Does the additional in-band traffic produces new attack surfaces?
	\item Which anomaly detection/evaluation method îs the best for BAS?
	\item Which data reduction is feasible and in which part of the system does it makes sense? (in the data collection agent)
\end{enumerate}
	
	\chapter{Building Automation Systems and Field Buses}
	% background about BAS and KNX
	\label{sec:background:bas}
	% !TeX spellcheck = en_GB

%\subsection{Introduction into Building Automation}
\label{sec:background:bas:intro}

\begin{comment}
		\begin{itemize}
		\item Increasing amount of complexity and requirement of comfort in private and commercial buildings \parencite{Merz2009}

		\item benefits regarding saving and managing energy \parencite{Merz2009}
		\item Ever changing requirements: "in commercial buildings flexibility is high on the agenda -- offices buildings, for example, should be designed in such way that they can be easily adapted to meet any change in use or requirements" \parencite{Merz2009}
		\item "In modern buildings there are variety of automation systems for heating, ventilating and air conditioning" \parencite{Merz2009}
		\item "control systems optimize energy consumption and enable support and maintenance personnel to carry out their jobs more efficiently" \parencite{Merz2009}
		
	\end{itemize}
\end{comment}

With the ever increasing amount of requirements and complexity with regards to monitoring and controlling lights, \gls{hvac}, and other aspects in private and commercial buildings, more versatile and flexible wiring and control solutions are required. 
Additionally, more flexible solutions offer possibilities like comfort functions, increased energy efficiency, and ease of maintenance. \parencite{Merz2009}
These functions are provided by modern \glsfirst{bas} and networks.
Especially in commercial buildings room requirements might change regularly, and hence systems in these buildings should be designed to accommodate such changes. \parencite{Merz2009} \todo{more?}

% ------------------------------------------------------------------------------
\subsection{KNX as an Example}
\label{sec:background:bas:knx}

%\alert{telegram vs frame. frame is used in the english standard document -> will use telegram}
\begin{comment}
	\begin{itemize}
		\item \knx or Konnex
		\item "formerly known as European Installation Bus (EIB)" \parencite{Merz2009}
		\item "designed to be used in electrical installations for implementing automated functions and processes in buildings" \parencite{Merz2009}
		\item "Different transmission media can be used for the bus: twisted-pair cable (KNX.TP), power line (KNX.PL), radio frequency (KNX.RF) and fiber-optic cable" \parencite{Merz2009}
		\item focus on KNX.TP
			\subitem 2 variants
			\subitem TP-0: 2400 Baud unbalanced, derived from BatiBUS \parencite{CENELEC2004} \parencite{DIN_EN_50090-5-2}
			\subitem TP-1: 9600 Baud balanced, derived from EIB (relevant here) \parencite{CENELEC2004} \parencite{DIN_EN_50090-5-2}
		\item KNX.PL and KNX.RF for integration into older buildings \parencite{Merz2009}
		\item Standardized as DIN EN 50090 written by \textcite{CENELEC2004} \parencite{DIN_EN_50090-5-2}
		\item "Large sections was approved for inclusion into ISO/IEC 14543" \parencite{Merz2009}
		\item "World wide only open standard for home and building control" \parencite{Merz2009}
		\item Benefits of \knx include
			\subitem more devices due to different manufactures
			\subitem large variety of devices (sensors, actuators, control/regulation, operation and monitoring)
	\end{itemize}
\end{comment}

One of the most commonly used systems for building automation systems in Germany and Europe is \Gls{knx} or Konnex. Formerly known as the \gls{eib}, \gls{knx} is ab association of multiple building automation vendors, co-operating in the KNX Association.\footnote{\url{https://www.knx.org/}}
It was designed to substitute traditional electrical installations and to implement functions and processes in an automated fashion. \parencite{Merz2009}

The \gls{knx} protocol is defined in the DIN EN 50090 family and can the transmitted over various physical media. Among those physical layers are \gls{knx} over twisted pair (KNX.TP), \gls{knx} over power line (KNX.PL) and \gls{knx} over radio (KNX.RF). Especially, KNX.PL and KNX.RF are intended for retrofitting older buildings.
Another more recent addition is KNX/IP, which transports the \gls{knx} physical layer over an \gls{ip} network and can be used to bridge multiple \gls{knx} lines, connect a central control unit, or to program \gls{knx} devices via the \gls{ets} software.
\parencite{Merz2009}

As a result, \gls{knx} twisted-pair is the most commonly used installation method and is defined in two flavours. \parencite{DIN_EN_50090-5-2}
The older variant TP-0 was derived from the BatiBUS \todo{ref?} and operates at 2400 Baud. However, nowadays most \gls{knx} devices implement TP-1, operating at 9600 Baud, which is a direct successor of \gls{eib}.

One of the most appealing benefits of using \gls{knx} is the ubiquitous availability as world wide accepted DIN standard, which was even amplified as large sections were included in ISO/IEC 14543.
Following, more devices from multiple vendors are available with a large variety of sensors, actuators, control and regulation solutions, as well as operation and monitoring.

From an academic point of view the easy availability of the standards, provides a good understanding of the protocol. Therefore allowing for analysis without prior reverse-engineering the protocol.
Additionally the University of Rostock also uses \gls{knx} in some buildings, which a provides an promising source of test and demo data. \parencite[cf.][]{Mundt2012}
Due to the ease of accessibility it is therefore used as an example for \gls{bas} in this thesis without the limitation of generality.

% ------------------------------------------------------------------------------
\subsubsection{Bus Topology of KNX Networks}
\label{sec:background:bas:knx:topo}

\begin{comment}
		\begin{itemize}
		\item "Like a conventional electrical installation, a KNX installation needs to have a power line network to provide the loads with electricity. But it also has a communication network – the KNX installation bus" \parencite{Merz2009}
		\item "Both networks are galvanically isolated" \parencite{Merz2009}
		\item 2 separate cable installations as a result
		\item \knx is designed to reshape the structure of conventional building installations, in form of a tree topology \parencite{Merz2009}
		\item "Nodes are assigned to a line" \parencite{Merz2009}
		\item "Several lines are connected to a main line and form an area" \parencite{Merz2009}
		\item "Several areas are connected with each other via the back bone line" \parencite{Merz2009}
		\item Each node is assigned an address, defining area, line, and device number which forms the logical topology
			\subitem cf. Table~\ref{tab:background:bas:knx:topo:addr}
		\item It is advised, that the physical topology reassembles the logical \parencite{Sokollik2017}
		\item Even though, the physical \knx bus can build in various forms (star, tree, and mixed forms) \parencite{Sokollik2017}
		\item Most significant subdivision of the physical topology in \knx.TP-1 is a line segment \parencite{Sokollik2017}
			\subitem a segment defines how many \knx devices can be connected on a physical line \parencite{Sokollik2017}
			\subitem segment is a part of the bus system, which connects \knx devices electrical continuous with each other. \parencite{Sokollik2017}
		\item 2 types of \knx.TP-1 devices \parencite{Sokollik2017}
			\subitem TP1-64 and TP1-256 cf. Table \ref{tab:background:bas:knx:topo:tpsegments}
			\subitem mainly differ in the maximum amount of devices, which can be connected in one segment
			\subitem problem since market is very fragmented
			\subitem products are often not clearly with regards to TP1-64 and TP1-256
			\subitem better of assuming TP1-64 for installations, since one TP1-64 device in a segment forces the whole segment to be downgraded to the TP1-64 standard
		\item therefore max a full logical line must be build in four segments with 3 line repeaters \parencite{Sokollik2017}
			\subitem it is to note that is not to advice since a \knx telegram can only take 6 hops
			\subitem e.g. can be transmitted by max 6 couplers
		\item area is created by coupling up to 16 lines \parencite{Sokollik2017}
			\subitem 1 main line and 15 sublines
			\subitem lines are coupled via line couplers to the main line
			\subitem line couplers have own physical address (device address 0) \parencite{Sokollik2017}
			\subitem line couplers galvanically separate main line and sub line, are capable of routing \parencite{Sokollik2017}
		\item all main lines are connected via backbone couplers to the backbone line \parencite{Merz2009}
	\end{itemize}
\end{comment}

To accommodate for the initially mentioned advanced features in building automation systems, like central control and management, improved energy efficiency, or other comfort functions \gls{knx} introduces an additional wiring installation (cf. Section~\ref{sec:background:bas:knx}) next to the conventional mains power line. Both of these installations are galvanically isolated and \gls{knx} merely use mains power for actuators e.g. lights or blinds, since the \gls{knx} devices are powered via the \gls{knx} bus from special \gls{knx} power supply units.
Each power supply is able to power up to 32 or 64 devices, depending on the model.
As a result, two separate cable installations are required.
Besides the physical \gls{knx} wiring can be constructed in various forms, like star, tree, or mixed forms, it is strongly advised to ensure that the physical wiring resembles the logical structure of the network. \parencite{Sokollik2017,Merz2009} \todo{visualise Data-Cables vs Topology}
\todo{write about more about physical installation of knx? -- only if reffed later}

\begin{figure}
	\centering
	\includegraphics[width=\textwidth]{figures/100-knx-demo-topo.pdf}
	\caption[KNX network topology]{Exemplary logical topology of a \gls{knx} network with line couplers (LC) and normal devices.}
	\label{fig:background:knx:topo}
\end{figure}

The logical structure builds on top of the physical topology and is intended to remodel the functional structure of buildings in a logical tree topology. This tree topology is divided into three levels: areas, lines, and devices. Each device or node is assigned a physical address, defining area, line, and device number. (cf. Table~\ref{tab:background:bas:knx:topo:addr})
Each device is directly connected to a line, which can accommodate up to 256 devices, if supported by all devices on the line. Alternatively, up to 64 devices can be in one line segment before line repeater is required. (cf. Table~\ref{tab:background:bas:knx:topo:tpsegments})
Most \gls{knx} devices still follow the TP1-64 specification. Due to the high market fragmentation devices are often not mark in regards of supporting either TP1-64 or TP1-256, therefore it is still recommended to only use 64 devices within a line segment. \alert{sentence too long?} Thus, a full logical line is advised to be build using three line repeater. \parencite{Sokollik2017,Merz2009}

\begin{table}
	\centering
	\begin{tabular}{l r r }
		& \textbf{TP1-64} & \textbf{TP1-256} \\\toprule
		maximum devices per segment & 64 & 256 \\
		maximum distance between 2 devices & 700m & 700m \\
		maximum length of a physical line & 1000m & 1000m \\
		\bottomrule
	\end{tabular}
	\caption[Segment properties for \knx.TP1-64 and \knx.TP1-256]{Segment properties for \knx.TP1-64 and \knx.TP1-256. cf. \textcite{Sokollik2017} \todo{adjust design of this table}}
	\label{tab:background:bas:knx:topo:tpsegments}
\end{table}

Usually a line coupler is installed as a line repeater. A line coupler is an active device to connect one logical level of the \gls{knx} network to another one, which can be one line segment to another line segment.
However, their original purpose is to separate the different levels of the logical \gls{knx} topology, explicitly to attach a line to the main line of an area or to attach a main line to the backbone. (cf. Figure~\ref{fig:background:knx:topo})
An area is formed by combining up to 16 lines: one main line and 15 sub-lines. All main lines are then connected to a backbone line, again via area couplers. \parencite{Sokollik2017}

As an active device line couplers are assigned an address, and are able to filter traffic by addresses similar to \gls{ip} router. %usually with device number 0 in case it connects a line to backbone line.
Further couplers provide galvanic isolation to mitigate damages by high voltages, shorted circuits, etc. As a result multiple power supplies are required, basically one for each line segment.
An important feature to be aware of, when using line couplers, is that a \gls{knx} telegram can only be routed over six hops. This was introduced to prevent high network load in case of cyclic routing, comparable to the time to live in IP based networks. \parencite{Sokollik2017}

% KNX Address byte-field

\begin{table}[b]
	\aboverulesep=0ex
	\belowrulesep=0ex
	\renewcommand{\arraystretch}{1.2}
	
	\centering
	\begin{tabular}{|l|c|c|c|c|c|c|c|c|c|c|c|c|c|c|c|c|}
		\toprule
		Byte & \multicolumn{8}{c|}{1} & \multicolumn{8}{c|}{0} \\\midrule
		Bit & 7 & 6 & 5 & 4 & 3 & 2 & 1 & 0 & 7 & 6 & 5 & 4 & 3 & 2 & 1 & 0\\\midrule
		Physical Address & \multicolumn{4}{c|}{area} & \multicolumn{4}{c|}{line} & \multicolumn{8}{c|}{device}\\\midrule
		Group Address & \multicolumn{5}{c|}{main} & \multicolumn{3}{c|}{middle} & \multicolumn{8}{c|}{sub group}\\
		\bottomrule
	\end{tabular}
	\caption[Bit field for \knx addresses]{Bit field for \knx addresses. Only the most common used 3-level group address is shown here. cf.~\textcite{Merz2009,Sokollik2017} }
	\label{tab:background:bas:knx:topo:addr}
\end{table}

Additional to the physical address each devices is assigned, devices may listen to group addresses.
\todo{describe layers/modes of group addresses?}
Whereby physical addresses are used to communicate with one specific device, which is common during installation, throughout normal operation mostly group addresses are used. As the name implies multiple devices may listen to such an address, allowing one light switch to turn on or off multiple lights for instance.

% ------------------------------------------------------------------------------	
\subsubsection{KNX Protocol}
\label{sec:background:bas:knx:proto}

\begin{comment}
		\begin{itemize}
		\item 3 types of telegrams \parencite{Hubner2009} \parencite{Merz2009}
			\subitem data - "sent in response to an individual action such as" \parencite{Merz2009} pressing a button (standard and extended \parencite{Hubner2009})
			\subitem acknowledge - "All devices (receivers) that belong to this group simultaneously confirm that they have received the data frame by returning an acknowledgment frame." \parencite{Merz2009}
			\subitem poll - "one device can query 1-Byte-Information from up to 15 other devices. Addressing is done via group addresses" \parencite{Hubner2009}
		\item Standard Data Telegram
			\subitem cf. Table~\ref{tab:background:bas:knx:proto:knx-standard}
		\item Extended Data Telegram
			\subitem cf. Table~\ref{tab:background:bas:knx:proto:knx-extended}
		\item Poll Telegram
			\subitem cf. Table~\ref{tab:background:bas:knx:proto:knx-poll}
		\item Acknowledge Telegram
			\subitem cf. Table~\ref{tab:background:bas:knx:proto:ack}
	\end{itemize}
\end{comment}

To communicate within a \gls{knx} network, devices send telegrams in their line. Due to the topology (cf. Section~\ref{sec:background:bas:knx:topo}) and technological nature of \gls{knx} these telegrams are always broadcasted within the respective line, only line couplers may narrow down the scope in which a specific telegram might be received. Telegrams are very comparable to packets in an IP network.

Within the \gls{knx} specification three types of telegrams are defined: Data Telegrams, Poll Telegrams, and Acknowledge Telegrams. (cf. Table~\ref{tab:background:bas:knx:proto:knx-standard},~\ref{tab:background:bas:knx:proto:ctrle},~\ref{tab:background:bas:knx:proto:knx-poll}, and~\ref{tab:background:bas:knx:proto:ack})
Each of which can be transmitted using on of four priorities, which are show in Table~\ref{tab:background:bas:knx:proto:prio}. Hereby it is to note, that a logical 0 is the dominant symbol. Consequently, the priority encoding is designed in a way, that lower priority telegrams are overruled by telegram with higher priority.
The bus devices detect these collisions and avoid them by following the \gls{csmaca} principle.

The telegram type is encoded in the first two bit of the \gls{ctrl} byte, which functions as a header of a \gls{knx} telegram. Next to the type it also contains a repeat flag, an acknowledge flag, and the priority. (see Figure~\ref{tab:background:bas:knx:proto:ctrl})

% KNX prio table

\begin{table}[h]
	\aboverulesep=0ex
	\belowrulesep=0ex
	\renewcommand{\arraystretch}{1.2}
	\newcolumntype{Y}{>{\centering\arraybackslash}X}
	
	\centering
	\begin{tabularx}{0.7\textwidth}{|l|Y|Y|}
		\toprule
		Bit & 1 & 0 \\\midrule
		Function & \multicolumn{2}{c|}{Priority} \\\bottomrule
		\toprule
		System priority & 0 & 0 \\\midrule
		Alarm priority & 1 & 0 \\\midrule
		High priority & 0 & 1 \\\midrule
		Low priority & 1 & 1 \\\bottomrule
	\end{tabularx}
	\caption[KNX priority encoding]{\acs{knx} priority encoding. cf.~\textcite{Merz2009}}
	\label{tab:background:bas:knx:proto:prio}
\end{table}

%\subsubsection{Data Telegram}
\paragraph{Data Telegram}
\label{sec:background:bas:knx:proto:data}

The data telegram being utilised the most and used for transferring commands, state, and value changes, comes in two flavours.
The standard data telegram (cf. Table~\ref{tab:background:bas:knx:proto:knx-standard}) is able to transmit between 2 and 16 bytes per telegram. To transfer more data the extended data telegram (cf. Table~\ref{tab:background:bas:knx:proto:knx-extended}) can be utilised, which can transfer up to 255 bytes.
Both are sent in reaction of an direct interaction (e.g. light switch) or to send periodical sensor data (e.g. temperature). \parencite{Merz2009}
Once received and successfully processed, the recipient replies then with an acknowledge telegram. \parencite{Merz2009,Sokollik2017}

The extended data telegram introduces another header byte after \gls{ctrl}, the \gls{ctrle} header byte. It contains information otherwise encoded together with the payload length like the destination address type and the hop count, thus it allows the payload length to encoded as a full byte, ranging up to 255. (see Table~\ref{tab:background:bas:knx:proto:ctrle})

% KNX data telegram structure for standard and extended

\begin{table}[p]
	\aboverulesep=0ex
	\belowrulesep=0ex
	\renewcommand{\arraystretch}{1.2}
	\newcolumntype{Y}{>{\centering\arraybackslash}X}
	
	\centering
	\begin{tabularx}{\textwidth}{|l|Y|Y|Y|Y|Y|Y|Y|Y|Y|Y|Y|Y|Y|Y|Y|Y|Y|Y|Y|Y|Y|Y|Y|Y|}
		\toprule
		Byte & \multicolumn{8}{c|}{0} & \multicolumn{8}{c|}{1} & \multicolumn{8}{c|}{2} \\\midrule
		Bit & & & & & & & & & & & & & & & & & & & & & & & & \\\midrule
		Function & \multicolumn{8}{c|}{CTRL} & \multicolumn{16}{c|}{Source Address} \\\bottomrule
		\toprule
		Byte & \multicolumn{8}{c|}{3} & \multicolumn{8}{c|}{4} & \multicolumn{8}{c|}{5} \\\midrule
		Bit & & & & & & & & & & & & & & & & & & & & & & & & \\\midrule
		Function & \multicolumn{16}{c|}{Destination Address} & \multicolumn{1}{c|}{\parbox[t][26pt][s]{0.2cm}{A T}} & \multicolumn{3}{c|}{Hops} & \multicolumn{4}{c|}{Length} \\\bottomrule
		\toprule
		Byte & \multicolumn{8}{c|}{6} & \multicolumn{8}{c|}{$7+n$} & \multicolumn{8}{c|}{$8+n$} \\\midrule
		Bit & & & & & & & & & & & & & & & & & & & & & & & & \\\midrule
		Function & \multicolumn{16}{c|}{Payload $n+1$ Bytes} & \multicolumn{8}{c|}{Parity} \\\bottomrule
	\end{tabularx}
	\caption[Standard KNX data telegram]{Standard \gls{knx} data telegram with $2$ to $16$ bytes of payload. Control Byte (CTRL) cf. Table~\ref{tab:background:bas:knx:proto:ctrl}, Source Address, Destination Address cf. Table~\ref{tab:background:bas:knx:topo:addr}, Address Type (AT), Hop Count (Hops), Payload Length (Length), Payload, and Parity.}
	\label{tab:background:bas:knx:proto:knx-standard}
\end{table}

\begin{table}[p]
	\aboverulesep=0ex
	\belowrulesep=0ex
	\renewcommand{\arraystretch}{1.2}
	\newcolumntype{Y}{>{\centering\arraybackslash}X}
	
	\centering
	\begin{tabularx}{\textwidth}{|l|Y|Y|Y|Y|Y|Y|Y|Y|Y|Y|Y|Y|Y|Y|Y|Y|Y|Y|Y|Y|Y|Y|Y|Y|}
		\toprule
		Byte & \multicolumn{8}{c|}{0} & \multicolumn{8}{c|}{1} & \multicolumn{8}{c|}{2} \\\midrule
		Bit & & & & & & & & & & & & & & & & & & & & & & & & \\\midrule
		Function & \multicolumn{8}{c|}{CTRL} & \multicolumn{8}{c|}{CTRLE} & \multicolumn{8}{c|}{Source Address} \\\bottomrule
		\toprule
		Byte & \multicolumn{8}{c|}{3} & \multicolumn{8}{c|}{4} & \multicolumn{8}{c|}{5} \\\midrule
		Bit & & & & & & & & & & & & & & & & & & & & & & & & \\\midrule
		Function & \multicolumn{8}{c|}{Source Address} & \multicolumn{16}{c|}{Destination Address} \\\bottomrule
		\toprule
		Byte & \multicolumn{8}{c|}{6} & \multicolumn{8}{c|}{$7$} & \multicolumn{8}{c|}{$8+n$} \\\midrule
		Bit & & & & & & & & & & & & & & & & & & & & & & & & \\\midrule
		Function & \multicolumn{8}{c|}{Length} & \multicolumn{16}{c|}{Payload $n+1$ Bytes} \\\bottomrule
		\toprule
		Byte & \multicolumn{8}{c|}{$9+n$} & \multicolumn{16}{c|}{ } \\\cmidrule{1-9}
		Bit & & & & & & & & & \multicolumn{16}{c|}{ } \\\cmidrule{1-9}
		Function & \multicolumn{8}{c|}{Parity} & \multicolumn{16}{c|}{ } \\\bottomrule
	\end{tabularx}
	\caption[Extended KNX data telegram]{Extended \gls{knx} data telegram with $2$ to $255$ bytes of payload. Control Byte (CTRL) cf. Table~\ref{tab:background:bas:knx:proto:ctrl}, Extended Control Byte (CTRLE) cf. Table~\ref{tab:background:bas:knx:proto:ctrle}, Source Address, Destination Address cf. Table~\ref{tab:background:bas:knx:topo:addr}, Payload Length (Length), Payload, and Parity.}
	\label{tab:background:bas:knx:proto:knx-extended}
\end{table}
% KNX telegram CTRL byte

\begin{table}
	\aboverulesep=0ex
	\belowrulesep=0ex
	\renewcommand{\arraystretch}{1.2}
	\newcolumntype{Y}{>{\centering\arraybackslash}X}
	
	\centering
	\begin{tabularx}{\textwidth}{|l|Y|Y|Y|Y|Y|Y|Y|Y|}
		\toprule
		Byte & \multicolumn{8}{c|}{0} \\\midrule
		Bit & 7 & 6 & 5 & 4 & 3 & 2 & 1 & 0 \\\midrule
		Function & \multicolumn{2}{c|}{TT} & R & A & \multicolumn{2}{c|}{P} & \multicolumn{2}{c|}{reserved} \\\bottomrule \toprule
		Standard Data Telegram & 1 & 0 & R & 1 & \multicolumn{2}{c|}{P} & \multicolumn{2}{c|}{ } \\\midrule
		Extended Data Telegram & 0 & 0 & R & 1 & \multicolumn{2}{c|}{P} & \multicolumn{2}{c|}{ } \\\midrule
		Poll Telegram & 1 & 1 & 1 & 1 & 0 & 0 & \multicolumn{2}{c|}{ } \\\midrule
		Acknowledge Telegram & \multicolumn{2}{c|}{ } & 0 &0 & \multicolumn{2}{c|}{ } & \multicolumn{2}{c|}{ } \\\bottomrule
	\end{tabularx}
	\caption[\knx CTRL Byte]{\knx CTRL Byte. Telegram Type (TT), Repeat (R), Acknowledge (A), and Priority (P). cf. \textcite{Sokollik2017}}
	\label{tab:background:bas:knx:proto:ctrl}
\end{table}
% KNX CTRL-Extended control byte

\begin{table}[p]
	\aboverulesep=0ex
	\belowrulesep=0ex
	\renewcommand{\arraystretch}{1.2}
	\newcolumntype{Y}{>{\centering\arraybackslash}X}
	
	\centering
	\begin{tabularx}{0.7\textwidth}{|l|Y|Y|Y|Y|Y|Y|Y|Y|}
		\toprule
		Byte & \multicolumn{8}{c|}{0} \\\midrule
		Bit & 7 & 6 & 5 & 4 & 3 & 2 & 1 & 0 \\\midrule
		Function & AT & \multicolumn{3}{c|}{Hops} & \multicolumn{4}{c|}{EFF} \\\bottomrule
	\end{tabularx}
	\caption[\knx CTRLE Byte]{\knx CTRLE Byte. Address Type (AT), Hop Count, and Extended Frame Format (EFF). cf. \textcite{Sokollik2017}}
	\label{tab:background:bas:knx:proto:ctrle}
\end{table}

%\subsubsection{Acknowledge Telegram}
\paragraph{Acknowledge Telegram}
\label{sec:background:bas:knx:proto:ack}

The acknowledge telegram can indicate the correct reception (\code{ACK}, positive acknowledge), faulty reception (\code{NACK}, negative acknowledge), or that the receiving device is currently busy and is not able to process the telegram. (cf. Table~\ref{tab:background:bas:knx:proto:ack})
It consists of one byte and is sent by the recipient after waiting 13 bit times.
If the sender receives an positive acknowledge, the telegram was send successfully and no further action is required. However, if it detects an negative acknowledge or busy signal, the telegram is resend up to three times.
Supposed a busy signal occurred, the sender adds a waiting period before attempting to resend the telegram.
Given, the sender does not receives any form of acknowledge telegram within 13 bit times, it resends the telegram also up to three times.
In case the sender addressed the original telegram to multiple devices by using a group address as destination, the acknowledge frame is send simultaneously by all recipients. Due to the bus characteristic of being normally high (1), an negative acknowledge or busy signal overrules a positive acknowledge, since \code{BUSY} and \code{NACK} are encoded with low bits (0). \parencite{Merz2009,Sokollik2017}

\begin{table}[p]
	\aboverulesep=0ex
	\belowrulesep=0ex
	\renewcommand{\arraystretch}{1.2}
	\newcolumntype{Y}{>{\centering\arraybackslash}X}
	
	\centering
	\begin{tabularx}{0.7\textwidth}{|l|Y|Y|Y|Y|Y|Y|Y|Y|}
		\toprule
		Byte & \multicolumn{8}{c|}{0} \\\midrule
		Bit & 7 & 6 & 5 & 4 & 3 & 2 & 1 & 0 \\\bottomrule
		\toprule
		ACK  & 1 & 1 & 0 & 0 & 1 & 1 & 0 & 0 \\\midrule
		NACK & 0 & 0 & 0 & 0 & 1 & 1 & 0 & 0 \\\midrule
		BUSY & 1 & 1 & 0 & 0 & 0 & 0 & 0 & 0 \\\midrule
		NACK + BUSY & 0 & 0 & 0 & 0 & 0 & 0 & 0 & 0 \\\bottomrule
	\end{tabularx}
	\caption[\knx acknowledge telegram]{\knx short acknowledge telegram.}
	\label{tab:background:bas:knx:proto:ack}
\end{table}

%\subsubsection{Poll Telegram}
\paragraph{Poll Telegram}
\label{sec:background:bas:knx:proto:poll}

The poll telegram allows to poll one byte of data from up to 15 devices, however, it is seldom used in \gls{knx} networks. It consists of a poll data request frame (cf. Table~\ref{tab:background:bas:knx:proto:knx-poll}) and a poll data response frame. Former starts the poll telegram and sets the group address to query, also it defines how many bytes to receive.
The poll data response consists of each device sending one byte on the bus without starting a new telegram. The responding devices are doing so in a prior configured time slot. \parencite{Hubner2009,DIN_EN_50090-5-2}

\begin{table}
	\aboverulesep=0ex
	\belowrulesep=0ex
	\renewcommand{\arraystretch}{1.2}
	\newcolumntype{Y}{>{\centering\arraybackslash}X}
	
	\centering
	\begin{tabularx}{\textwidth}{|l|Y|Y|Y|Y|Y|Y|Y|Y|Y|Y|Y|Y|Y|Y|Y|Y|Y|Y|Y|Y|Y|Y|Y|Y|}
		\toprule
		Byte & \multicolumn{8}{c|}{0} & \multicolumn{8}{c|}{1} & \multicolumn{8}{c|}{2} \\\midrule
		Bit & & & & & & & & & & & & & & & & & & & & & & & & \\\midrule
		Function & \multicolumn{8}{c|}{CTRL} & \multicolumn{16}{c|}{Source Address} \\\bottomrule
		\toprule
		Byte & \multicolumn{8}{c|}{3} & \multicolumn{8}{c|}{4} & \multicolumn{8}{c|}{5} \\\midrule
		Bit & & & & & & & & & & & & & & & & & & & & & & & & \\\midrule
		Function & \multicolumn{16}{c|}{Destination Address} & \multicolumn{4}{c|}{ } & \multicolumn{4}{c|}{poll data} \\\bottomrule
		\toprule
		Byte & \multicolumn{8}{c|}{6} & \multicolumn{16}{c|}{ } \\\cmidrule{1-9}
		Bit & & & & & & & & & \multicolumn{16}{c|}{ } \\\cmidrule{1-9}
		Function & \multicolumn{8}{c|}{Parity} & \multicolumn{16}{c|}{ } \\\bottomrule
	\end{tabularx}
	\caption[\knx poll telegram]{\knx poll telegram. Control Byte (CTRL) cf. Table~\ref{tab:background:bas:knx:proto:ctrl}, Source Address, Destination Address cf. Table~\ref{tab:background:bas:knx:topo:addr}, expected length of poll data (poll data), and Parity.}
	\label{tab:background:bas:knx:proto:knx-poll}
\end{table}

\subsubsection{KNX Communication}
\label{sec:background:bas:knx:communication}

\begin{comment}
		\begin{itemize}
		\item application layer described in \textcite{DIN_EN_50090-4-1}
		\item communication modes \parencite{DIN_EN_50090-4-1}
			\subitem point-to-multipoint, connection-less (multicast)
			\subitem point-to-domain, connection-less (broadcast)
			\subitem point-to-all-points, connection-less (system broadcast)
			\subitem point-to-point, connection-less
			\subitem point-to-point, connection-oriented
		\item APCI describes operation on application layer
		\item Response need to match the Request APCI type and must be send on the corresponding communication mode allowed for this packet. cf. \textcite[][page 9--10]{DIN_EN_50090-4-1}
	\end{itemize}
\end{comment}

Apart from the basic \gls{knx} protocol (cf. Section~\ref{sec:background:bas:knx:proto}), which is comparable to the data link layer in the \gls{osi} model, the \textcite{DIN_EN_50090-4-1} describes two layers on top: a transport layer and an application layer.
However, it is to note, that even when official \gls{knx} specification draws parallels to the \gls{osi} model, a direct comparison to the \gls{osi} model as seen in \gls{ip} networks is difficult, due to the reduced complexity of the protocol stack in \gls{knx}.

The transport layer defines multiple transport layer services, which are encoded in the \gls{tpci} (cf. Table~\ref{tab:background:bas:knx:comm:tpci-apci}) and can be clustered into five categories or modes:

\begin{enumerate}
	\item point-to-multipoint, connection-less (multicast)
	\item point-to-domain, connection-less (broadcast)
	\item point-to-all-points, connection-less (system broadcast)
	\item point-to-point, connection-less
	\item point-to-point, connection-oriented
\end{enumerate}
\parencite[p.~30]{DIN_EN_50090-4-2}

Depending on the category, multiple telegrams might be send in order to establish a connection, e.g. point-to-point, connection oriented.
Further, these communication modes are used by the \alert{54} different application layer services described in \textcite{DIN_EN_50090-4-1}.
These application layer service can be used to access and set device states, encoded in \glspl{dpt}, or to configure the device.
Each response needs to match the requests \gls{apci} type and must be send on with the corresponding communication mode allowed for this telegram. \parencite[pp.~9--10]{DIN_EN_50090-4-1}

\glspl{dpt}, in turn, define the structure of the data, which is transported using the application layer services.
They are defined to consist of a combination of a data type and a dimension. Whereby the former is required to specify format and encoding. The later, however, defines range and unit.
It was decided not to specify data types and dimensions separately to reduce abstract naming and confusion.
Therefore any \gls{dpt} declares a single combination of format, encoding, range, and unit. \parencite[p.~38]{DIN_EN_50090-3-3}

Each \gls{dpt} is identified by an 16 bit main number and an 16 bit sub-number, separated by a dot. The main number describes the format and encoding and consequently the sub-number describes range and unit. As a result, \glspl{dpt} with the same main number share the format and encoding.

\begin{table}
	\aboverulesep=0ex
	\belowrulesep=0ex
	\renewcommand{\arraystretch}{1.2}
	\newcolumntype{Y}{>{\centering\arraybackslash}X}
	
	\centering
	\begin{tabularx}{\textwidth}{|l|Y|Y|Y|Y|Y|Y|Y|Y|Y|Y|Y|Y|Y|Y|Y|Y|}
		\toprule
		Byte & \multicolumn{8}{c|}{6} & \multicolumn{8}{c|}{7} \\\midrule
		Bit & & & & & & & & & & & & & & & & \\\midrule
		\multirow{2}{*}{Function} & \multicolumn{6}{c|}{TPCI} & \multicolumn{10}{c|}{APCI} \\\cmidrule{2-7}
		 & DC & No & \multicolumn{4}{c|}{Seq. Number} & \multicolumn{10}{c|}{} \\\bottomrule
	\end{tabularx}

	\caption[KNX TPCI/APCI structure in a standard data telegram]{\gls{knx} \acrshort{tpci}/\acrshort{apci} structure in a standard data telegram. Data/Control flag (DC), Numbered/Unnumbered flag (No), Sequence Number (Seq. Number).}
	\label{tab:background:bas:knx:comm:tpci-apci}
\end{table}
	
\subsubsection{KNX Management Procedures}
\label{sec:background:bas:knx:management}

\begin{comment}
	\begin{itemize}
		\item "describe the device independent management procedures" \parencite{DIN_EN_50090-7-1}
		\item "shall be used to configure the network, and to get the information about the configuration of the network and connected devices" \parencite{DIN_EN_50090-7-1}
		\item " no knowledge of the single devices is required. They will work with every device connected to the network" \parencite{DIN_EN_50090-7-1}
		\item "shall be based on the use of the dedicated application layer services which are specified in EN 50090-4-1 \parencite{DIN_EN_50090-4-1} for this purpose" \parencite{DIN_EN_50090-7-1}
	\end{itemize}
\end{comment}

To ensure maximum compatibility among \gls{knx} devices of different vendors, it is not enough to implement the same protocol stack (cf. Section~\ref{sec:background:bas:knx:proto}) and use the same communication services (cf. Section~\ref{sec:background:bas:knx:communication}).
Moreover, it is also required to share a common set of management and configuration services.
This is especially true in an environment like \gls{knx}, where devices need to be fully interchangeable and need to be configurable by a common tool used by electricians.

Therefore the \gls{knx-assoc} specified two sets of management procedure.
For these procedures dedicated application layer services were defined in \textcite{DIN_EN_50090-4-1}. (cf. Section~\ref{sec:background:bas:knx:communication})
The first, being called network management procedures, are device independent, which do not require any knowledge or insight of the devices and will work across all vendors and device types. \parencite[pp.~11~ff.]{DIN_EN_50090-7-1}
They mainly provide routines to configure addresses, read serial numbers, get functions, or to find all devices in programming mode.

Secondly, there are device management procedures, which require in depth knowledge of the devices. \parencite[pp.~30~ff.]{DIN_EN_50090-7-1}
These are used to i.e. configure devices, restart them, and verify the configuration.

\subsubsection{Security in KNX}
\label{sec:background:bas:knx:security}

\begin{comment}
	\begin{itemize}
		\item copulers allow filtering 
			\subitem "This means that data frames transmitted by a sender are only forwarded to recipients that are not on the sender’s line." \parencite{Merz2009}
			\subitem "The filter function only forwards data frames to where they are needed. This reduces the overall amount of data frame traffic and keeps the data traffic on one line separate from the data traffic on another line, allowing data to be transferred on several lines simultaneously." \parencite{Merz2009}
		\item main group 14 and 15 should not be used \parencite{Hubner2009}
			\subitem can not be filtered due to limited space for filter tables in couplers \parencite{Hubner2009}
			
		\item cf DIN50090-3-4 page 13 (secure application layer) \parencite{DIN_EN_50090-3-4}
	\end{itemize}
\end{comment}

With \gls{knx} being so widely used, it is also deployed in security sensitive areas like access control\footnote{\url{http://search-ext.abb.com/library/Download.aspx?DocumentID=2CDC500074M0201&LanguageCode=en&DocumentPartId=&Action=Launch}}, ventilation, or fire alarms\footnote{\url{https://www.knx.org/media/docs/downloads/Marketing/Flyers/KNX-Solutions/KNX-Solutions_en.pdf}}.
As a consequence it is only sensible to consider not only the operational security, but also the information security of \gls{knx} networks. Meaning, it should not only be assured, that the network is operational at any time, but it needs to be protected against malicious usage or intrusion.

By design \gls{knx} introduces a zone concept by distinguishing between lines and areas.
This offers the possibility to limit a potential attack to just one line, because the traffic between lines is filtered by line couplers.
This filter function ensures, that only traffic concerning devices in other lines is routed to the main or backbone line.

However, filtering traffic comes with certain limitations. For instance the line couplers need be configured properly, which can be done automatically by the installation and configuration Software \gls{ets}. \parencite{Merz2009}
Further, to ensure proper filtering the main groups 14 and 15 must not be used, since they cannot be filtered, due to limited space for filter tables in couplers. \parencite{Hubner2009}
Moreover, traffic within the affected line can still be red and manipulated by an attacker, as well as device can be reconfigured or attacked.

Another possibility to secure a \gls{knx} network is to enable \gls{s-al}, as defined in \textcite{DIN_EN_50090-3-4}. \Gls{s-al} is an extension of the \gls{knx} specification defining an encrypted application layer (cf. Section~\ref{sec:background:bas:knx:communication}) for multicast and point-to-point communication, which can be used next to normal unencrypted traffic.
As encryption standard \textcite{DIN_EN_50090-3-4} defines \gls{aes}-128 in \gls{ctr} or \gls{ccm} mode.
Later also functions as authentication. \todo{ensure short formes are used, at least for AES}

Nonetheless it is to note, that \textcite{DIN_EN_50090-3-4} is merely a draft at the point of June 2017, therefore it is not to expect, that available \gls{knx} devices with this capability will be on the market any time soon. This may also be caused by the relatively high life time of devices which are installed permanently in buildings.


% ------------------------------------------------------------------------------
%\section{\lonworks}

%\section{\bacnet}
	
	\chapter{Network Analysis and Netflow}
	% background about netflow
	\label{sec:background:network}
	% !TeX spellcheck = en_US

\section{General Network Monitoring?}

\section{Anomaly Detection?}

\section{Netflow}
\begin{itemize}
	\item active
		\begin{itemize}
			\item inject additional traffic to measure things \parencite{Hofstede2014}
			\item e.g. ping, traceroute
		\end{itemize}
	\item passive
		\begin{itemize}
			\item "observe existing traffic as it passes by" \parencite{Hofstede2014}
			\item e.g. packet capture "This method generally provides most insight into the network traffic, as complete packets can be captured and further analyzed. However, in high-speed networks with line rates of up to 100 Gbps, packet capture requires expensive hardware and substantial infrastructure for storage and analysis." \parencite{Hofstede2014}
			\item "Another passive network monitoring approach that is more scalable for use in high-speed networks is flow export, in which packets are aggregated into flows and exported for storage and analysis." \parencite{Hofstede2014}
			\item  "a set of IP packets passing an observation point in the network during a certain time interval, such that all packets belonging to a particular flow have a set of common properties." \parencite{Claise2013}
			
		\end{itemize}
	\item 
\end{itemize}

\section{Intrusion Detection Systems (IDS)}
	\subsection{Signature Intrusion Detection Systems (SIDS)}
	\subsection{Anomaly-based Intrusion Detection Systems (ABIDS)}
	\subsection{State-based Intrusion Detection Systems\hint{??}}

\section{Anomaly and Novelty Detection}

	\subsection{General Approaches}
	\begin{itemize}
		\item cf. \textcite{Hodge2004}
		\item Type 1: "Determine the outliers with no prior knowledge of the data" \parencite{Hodge2004}
		\subitem "unsupervised clustering" \parencite{Hodge2004}
		\subitem "assumes that errors or faults are separated from the ‘normal’ data and will thus appear as outliers" \parencite{Hodge2004}
		\subitem " predominantly retrospective and is analogous to a batch-processing system" \parencite{Hodge2004} - Requires a bunch of data to begin with
		\subitem " requires that all data be available before processing" \parencite{Hodge2004}
		\subitem "data is static" \parencite{Hodge2004}
		\subitem "two sub-techniques" \parencite{Hodge2004}
		\subitem -> "outlier diagnostic approach highlights the potential outlying points" \parencite{Hodge2004} and "prune the outliers and fit their system model to the remaining data until no more outliers are detected" \parencite{Hodge2004}
		\subitem -> "accommodation that incorporates the outliers into the distribution model generated and employs a robust classification method" \parencite{Hodge2004} which "can withstand outliers in the data and generally induce a boundary of normality around the majority of the data" \parencite{Hodge2004}
		\subitem "Non-robust methods are best suited when there are only a few outliers in the data set" \parencite{Hodge2004}
		\subitem "a robust method must be used if there are a large number of outliers to prevent this distortion" \parencite{Hodge2004}
		\item Type 2: "Model both normality and abnormality." \parencite{Hodge2004}
		\subitem "supervised classification" \parencite{Hodge2004}
		\subitem "requires pre-labelled data" \parencite{Hodge2004}
		\subitem "The entire area outside the normal class represents the outlier class" \parencite{Hodge2004}
		\subitem "best suited to static data" \parencite{Hodge2004} "classification needs to be rebuilt from first principles if the data distribution shifts" \parencite{Hodge2004}
		\subitem "can be used for on-line classification" \parencite{Hodge2004} learning while new samples are classified
		\subitem "require a good spread of both normal and abnormal data" \parencite{Hodge2004}
		\subitem "classification is limited to a ‘known’ distribution" \parencite{Hodge2004}
		\subitem "cannot always handle outliers from unexpected regions" \parencite{Hodge2004}
		\item Type 3: "Model only normality [...]" \parencite{Hodge2004}
		\subitem "novelty detection" \parencite{Hodge2004} " semi-supervised recognition or detection task" \parencite{Hodge2004}
		\subitem "needs pre-classified data but only learns data marked normal" \parencite{Hodge2004}
		\subitem "suitable for static or dynamic data" \parencite{Hodge2004}
		\subitem "can learn the model incrementally as new data arrives" \parencite{Hodge2004}
		\subitem "aims to define a boundary of normality" \parencite{Hodge2004}
		\subitem "requires the full gamut of normality to be available for training to permit generalisation" \parencite{Hodge2004} but "requires no abnormal data" \parencite{Hodge2004}
		\subitem this is good, because "Abnormal data is often difficult to obtain or expensive in many fault detection domains [...]" \parencite{Hodge2004}
		\subitem "as long as the new fraud lies outside the boundary of normality then the system will be correctly detect the fraud" \parencite{Hodge2004}
		
	\end{itemize}
	
	\subsection{Anomaly Detection Methods}
	
	\subsubsection{Local Outlier Factor}
	\begin{itemize}
		\item seems to be a good fit
		\item works on unclean training data
		\item good selection of vectors and distance functions is required
		\item 11\% of the first 500 telegrams in \verb|eiblog.txt| are considered outliers
		\item density based
	\end{itemize}
	
	\subsubsection{On Class SVM}
	\begin{itemize}
		\item aka. novelty detection
		\item all training data is considered good
	\end{itemize}
	
	\subsubsection{Isolation Forest}
	\begin{itemize}
		\item ...
	\end{itemize}
	
	\subsubsection{Elliptical Envelope}
	\begin{itemize}
		\item tries to fit data to estimated \emph{shape}
		\item does not seem to be a good match, since it required to know the distribution of the vector fields
	\end{itemize}

	\subsection{Statistical BlaBla}
	\subsection{Entropy based}
	\subsection{Bayesian}
	\subsection{Pattern Matching}
	\subsection{Autoassociative Kernel Regression (AAKR)}
		used by \textcite{Yang2006}

\section{Time-based Anomaly Detection}

\section{Prior Work and Existing Solutions}
\hint{as own chapter?}
\begin{itemize}
	\item \parencite{Yang2006}
	\item \parencite{Celeda2012}
	\item \textcite{Pan2014} BACnet
		\begin{itemize}
			\item Anonmaly detection for BACnet fire alarm systems
			\item taps BACnet traffic via IP
			\item rule based learning (RIPPER)
			\item protection against common BACnet attack vectors
				\subitem who-is/who-has network probing
				\subitem write-property (take control over device)
				\subitem InitializedRoutingTable
				\subitem reinitialize devices
				\subitem application layer DoS
				\subitem flooding of network
		\end{itemize}
\end{itemize}
	
	\chapter{Conceptual Architecture}
	% concept
	\label{sec:concept}
	% !TeX spellcheck = en_US

\section{Design of Netflow Agent}
	\subsection{Truncating and Statistic Gathering}

\section{Monitor Pipeline}

\section{Anomaly detection}

	\subsection{Anomaly Detection Methods}
	
	\subsubsection{Approaches}
		\begin{itemize}
			\item cf. \textcite{Hodge2004}
			\item Type 1: "Determine the outliers with no prior knowledge of the data" \parencite{Hodge2004}
				\subitem "unsupervised clustering" \parencite{Hodge2004}
				\subitem "assumes that errors or faults are separated from the ‘normal’ data and will thus appear as outliers" \parencite{Hodge2004}
				\subitem " predominantly retrospective and is analogous to a batch-processing system" \parencite{Hodge2004} - Requires a bunch of data to begin with
				\subitem " requires that all data be available before processing" \parencite{Hodge2004}
				\subitem "data is static" \parencite{Hodge2004}
				\subitem "two sub-techniques" \parencite{Hodge2004}
				\subitem -> "outlier diagnostic approach highlights the potential outlying points" \parencite{Hodge2004} and "prune the outliers and fit their system model to the remaining data until no more outliers are detected" \parencite{Hodge2004}
				\subitem -> "accommodation that incorporates the outliers into the distribution model generated and employs a robust classification method" \parencite{Hodge2004} which "can withstand outliers in the data and generally induce a boundary of normality around the majority of the data" \parencite{Hodge2004}
				\subitem "Non-robust methods are best suited when there are only a few outliers in the data set" \parencite{Hodge2004}
				\subitem "a robust method must be used if there are a large number of outliers to prevent this distortion" \parencite{Hodge2004}
			\item Type 2: "Model both normality and abnormality." \parencite{Hodge2004}
				\subitem "supervised classification" \parencite{Hodge2004}
				\subitem "requires pre-labelled data" \parencite{Hodge2004}
				\subitem "The entire area outside the normal class represents the outlier class" \parencite{Hodge2004}
				\subitem "best suited to static data" \parencite{Hodge2004} "classification needs to be rebuilt from first principles if the data distribution shifts" \parencite{Hodge2004}
				\subitem "can be used for on-line classification" \parencite{Hodge2004} learning while new samples are classified
				\subitem "require a good spread of both normal and abnormal data" \parencite{Hodge2004}
				\subitem "classification is limited to a ‘known’ distribution" \parencite{Hodge2004}
				\subitem "cannot always handle outliers from unexpected regions" \parencite{Hodge2004}
			\item Type 3: "Model only normality [...]" \parencite{Hodge2004}
				\subitem "novelty detection" \parencite{Hodge2004} " semi-supervised recognition or detection task" \parencite{Hodge2004}
				\subitem "needs pre-classified data but only learns data marked normal" \parencite{Hodge2004}
				\subitem "suitable for static or dynamic data" \parencite{Hodge2004}
				\subitem "can learn the model incrementally as new data arrives" \parencite{Hodge2004}
				\subitem "aims to define a boundary of normality" \parencite{Hodge2004}
				\subitem "requires the full gamut of normality to be available for training to permit generalisation" \parencite{Hodge2004} but "requires no abnormal data" \parencite{Hodge2004}
				\subitem this is good, because "Abnormal data is often difficult to obtain or expensive in many fault detection domains [...]" \parencite{Hodge2004}
				\subitem "as long as the new fraud lies outside the boundary of normality then the system will be correctly detect the fraud" \parencite{Hodge2004}
				
		\end{itemize}
	
	\subsubsection{Local Outlier Factor}
		\begin{itemize}
			\item seems to be a good fit
			\item works on unclean training data
			\item good selection of vectors and distance functions is required
			\item 11\% of the first 500 telegrams in \verb|eiblog.txt| are considered outliers
			\item density based
		\end{itemize}
	
	\subsubsection{On Class SVM}
		\begin{itemize}
			\item aka. novelty detection
			\item all training data is considered good
		\end{itemize}
	
	\subsubsection{Isolation Forest}
		\begin{itemize}
			\item ...
		\end{itemize}
	
	\subsubsection{Elliptical Envelope}
		\begin{itemize}
			\item tries to fit data to estimated \emph{shape}
			\item does not seem to be a good match, since it required to know the distribution of the vector fields
		\end{itemize}
	
	\chapter{Results}
	% ...
	
	\chapter{Prototype Implementation}
	
	\chapter{Discussion}
	% discussion (duh)
	
	
	% --------------------------------------------------------------------------
	% statement of own work
	\newpage
	\pagenumbering{gobble}  % turn off page numbering for this page
	
	\chapter*{Selbstständigkeitserklärung}
	Hiermit erkläre ich Martin Peters, dass ich die vorliegende Bachelorarbeit selbständig angefertigt habe. Es wurden nur die in der Arbeit ausdrücklich benannten Quellen und Hilfsmittel benutzt. Wörtlich oder sinngemäß übernommenes Gedankengut habe ich als solches kenntlich gemacht.
	\\[64pt]
	\noindent Rostock, \thedate
	\begin{flushright}
		$\overline{~~~~~~~~~\mbox{(Unterschrift)}~~~~~~~~~}$
	\end{flushright}
	
	% restart page numbering
	\newpage
	\pagenumbering{Roman}
	
	% --------------------------------------------------------------------------
	% Bibliography
	%\nocite{*}
	\printbibliography
	
	% --------------------------------------------------------------------------
	% Glossary
	\newpage
	\printglossary[type=\acronymtype]
	%\newpage
	\printglossary[type=main]
	%\printglossaries
	
	% --------------------------------------------------------------------------
	% Appendix
	\begin{appendices}
		\noappendicestocpagenum
		\addappheadtotoc
		
		%\chapter*{Appendix}
		\label{sec:appendix}
		% !TeX spellcheck = en_GB
% ------------------------------------------------------------------------------
% code highlighting
% some nice styling for code listings
\definecolor{mygreen}{rgb}{0,0.6,0}
\definecolor{mygray}{rgb}{0.5,0.5,0.5}
\definecolor{mymauve}{rgb}{0.58,0,0.82}
\definecolor{lightgray}{rgb}{0.97,0.97,0.97}
%
\lstset{ %
	backgroundcolor=\color{lightgray},   % choose the background color; you must add \usepackage{color} or \usepackage{xcolor}
	basicstyle=\footnotesize,        % the size of the fonts that are used for the code
	breakatwhitespace=false,         % sets if automatic breaks should only happen at whitespace
	breaklines=true,                 % sets automatic line breaking
	captionpos=b,                    % sets the caption-position to bottom
	commentstyle=\color{mygreen},    % comment style
	deletekeywords={...},            % if you want to delete keywords from the given language
	escapeinside={\%*}{*)},          % if you want to add LaTeX within your code
	extendedchars=true,              % lets you use non-ASCII characters; for 8-bits encodings only, does not work with UTF-8
	frame=single,	                   % adds a frame around the code
	framerule=0pt,
	keepspaces=true,                 % keeps spaces in text, useful for keeping indentation of code (possibly needs columns=flexible)
	%keywordstyle=\color{blue},       % keyword style
	%keywordstyle={},
	language=Octave,                 % the language of the code
	otherkeywords={*,...},           % if you want to add more keywords to the set
	numbers=left,                    % where to put the line-numbers; possible values are (none, left, right)
	numbersep=5pt,                   % how far the line-numbers are from the code
	numberstyle=\tiny\color{mygray}, % the style that is used for the line-numbers
	rulecolor=\color{black},         % if not set, the frame-color may be changed on line-breaks within not-black text (e.g. comments (green here))
	showspaces=false,                % show spaces everywhere adding particular underscores; it overrides 'showstringspaces'
	showstringspaces=false,          % underline spaces within strings only
	showtabs=false,                  % show tabs within strings adding particular underscores
	stepnumber=2,                    % the step between two line-numbers. If it's 1, each line will be numbered
	%stringstyle=\color{mymauve},     % string literal style
	stringstyle=\textit,
	tabsize=2,	                   % sets default tabsize to 2 spaces
	title=\lstname                   % show the filename of files included with \lstinputlisting; also try caption instead of title
}
%
% ------------------------------------------------------------------------------

\chapter{Additional Visualisations of Anomaly-Detection Metrics}
\chaptermark{Additional Visualisations}
\label{app:metrics}
\newpage

\section{Metrics During the Training Phase}
\label{app:metrics:training}

\begin{figure}[H]
	\newcommand{\figwith}{0.49\textwidth}
	\newcommand{\figprefix}{train}
	\centering
	
	\begin{subfigure}[b]{\figwith}
		\includegraphics[width=\textwidth]{figures/700-results/\figprefix-tc.png}
		\caption{Telegrams (green) and telegrams with unknown addresses (yellow) over time}
		\label{fig:results:\figprefix:tc}
	\end{subfigure}
	\hfil
	\begin{subfigure}[b]{\figwith}
		\includegraphics[width=\textwidth]{figures/700-results/\figprefix-unknown-addr.png}
		\caption{Amount of observed unknown source (green) and destination (yellow) addresses}
		\label{fig:results:\figprefix:addr}
	\end{subfigure}
	\\[1.5mm]
	\begin{subfigure}[b]{\figwith}
		\includegraphics[width=\textwidth]{figures/700-results/\figprefix-num-outliers.png}
		\caption{Number of detected outliers via SVM (red) and LOF (blue).}
		\label{fig:results:\figprefix:outlier}
	\end{subfigure}
	\hfil
	\begin{subfigure}[b]{\figwith}
		\includegraphics[width=\textwidth]{figures/700-results/\figprefix-entropy.png}
		\caption{Calculated Entropy over time, infinity is encoded as $100 000$}
		\label{fig:results:\figprefix:entropy}
	\end{subfigure}
	\\[1.5mm]
	\begin{subfigure}[b]{\figwith}
		\includegraphics[width=\textwidth]{figures/700-results/\figprefix-lof-spread.png}
		\caption{Spread and average of the Local Outlier Factor, smaller numbers indicate outlier}
		\label{fig:results:\figprefix:lof}
	\end{subfigure}
	\hfil
	\begin{subfigure}[b]{\figwith}
		\includegraphics[width=\textwidth]{figures/700-results/\figprefix-svm-spread.png}
		\caption{Spread and average of the distance to decision surface, negative distances are outlier}
		\label{fig:results:\figprefix:svm}
	\end{subfigure}
	
	\caption[Detection Results During the Training Phase]{Detection Results During the Training Phase from 2017-01-21 00:00 to 2017-02-04 23:59 with a minimal time resolution of 1 hour.}
	\label{fig:results:\figprefix}
	
\end{figure}

\section{Metrics During the Validation Phase}
\label{app:metrics:validation}

\begin{figure}[H]
	\newcommand{\figwith}{0.49\textwidth}
	\newcommand{\figprefix}{validation}
	\centering
	
	\begin{subfigure}[b]{\figwith}
		\includegraphics[width=\textwidth]{figures/700-results/\figprefix-tc.png}
		\caption{Telegrams (green) and telegrams with unknown addresses (yellow) over time}
		\label{fig:results:\figprefix:tc}
	\end{subfigure}
	\hfil
	\begin{subfigure}[b]{\figwith}
		\includegraphics[width=\textwidth]{figures/700-results/\figprefix-unknown-addr.png}
		\caption{Amount of observed unknown source (green) and destination (yellow) addresses}
		\label{fig:results:\figprefix:addr}
	\end{subfigure}
	\\[1.5mm]
	\begin{subfigure}[b]{\figwith}
		\includegraphics[width=\textwidth]{figures/700-results/\figprefix-num-outliers.png}
		\caption{Number of detected outliers via SVM (red) and LOF (blue).}
		\label{fig:results:\figprefix:outlier}
	\end{subfigure}
	\hfil
	\begin{subfigure}[b]{\figwith}
		\includegraphics[width=\textwidth]{figures/700-results/\figprefix-entropy.png}
		\caption{Calculated Entropy over time, infinity is encoded as $100 000$}
		\label{fig:results:\figprefix:entropy}
	\end{subfigure}
	\\[1.5mm]
	\begin{subfigure}[b]{\figwith}
		\includegraphics[width=\textwidth]{figures/700-results/\figprefix-lof-spread.png}
		\caption{Spread and average of the Local Outlier Factor, smaller numbers indicate outlier}
		\label{fig:results:\figprefix:lof}
	\end{subfigure}
	\hfil
	\begin{subfigure}[b]{\figwith}
		\includegraphics[width=\textwidth]{figures/700-results/\figprefix-svm-spread.png}
		\caption{Spread and average of the distance to decision surface, negative distances are outlier}
		\label{fig:results:\figprefix:svm}
	\end{subfigure}
	
	\caption[Detection Results During the Validation Phase]{Detection Results During the Validation Phase from 2017-02-04 00:00 to 2017-02-11 23:59 with a minimal time resolution of 1 hour.}
	\label{fig:results:\figprefix}
	
\end{figure}

\chapter{Steps to Reproduce the Experiment Results}
\chaptermark{Reproduction Steps}
\label{app:reproduce}

Before attempting to build the sources, please ensure that following software packages are installed and working:
\code{Git}, \code{Docker 17.05.0-ce}, \code{docker-compose 1.19.0}, \code{Python 3.6.4+}, \code{Virtualenv 13.1.2}, and \code{pip 9.0.1}.

All described steps were developed and tested on
\code{Linux 4.14.0-3-amd64 \#1 SMP Debian 4.14.17-1 (2018-02-14) x86\_64 GNU/Linux}.\\
Alternative platforms might need adjustments to the build scripts or steps.

% set some new settings for listings
\lstset{
	language=Bash,
	breaklines=true,
	breakatwhitespace=true,
	numbers=none,
	frame=single,
	framerule=0pt,
	belowskip=-24pt,
	aboveskip=5pt,
	basicstyle=\footnotesize\ttfamily,
	stringstyle={},
	keywordstyle={},
}

\section{Build and Install the Prototype}
\label{app:reproduce:build}

\begin{enumerate}
	\item Clone the git repository or alternatively use the source repository on the attached disk in Appendix~\ref{app:disk}
		\begin{enumerate}
			\item Clone the main repository
\begin{lstlisting}
git clone https://github.com/FreakyBytes/master-thesis.git
\end{lstlisting}
			\item Navigate into the newly created directory
\begin{lstlisting}
cd master-thesis
\end{lstlisting}			
			\item Clone all submodules
\begin{lstlisting}
git submodule init
git submodule update
\end{lstlisting}
		\end{enumerate}
	\item Create a temporary Python 3 virtual environment, the directory should be changed for more permanent installations
\begin{lstlisting}
virtualenv -p $(which python3) /tmp/bob-venv
\end{lstlisting}
	\item Activate the virtual environment. \textbf{This step must be repeated whenever a new terminal session is started.}
\begin{lstlisting}
source /tmp/bob-venv/bin/activate
\end{lstlisting}
	\item Install the \gls{baos} parser library
\begin{lstlisting}
pip install --process-dependency-links --editable src/baos-knx-parser
\end{lstlisting}
	\item Install the \code{BOb} command line tool
\begin{lstlisting}
pip install --process-dependency-links --editable src/bas-observe
\end{lstlisting}
	\item Check if everything was installed correctly
\begin{lstlisting}
bob --help
\end{lstlisting}
\end{enumerate}

\section{Setup Required Services}
\label{app:reproduce:services}

\begin{enumerate}
	\item Start a new terminal session, not using the virtual environment
	\item Ensure the docker daemon is up and running
\begin{lstlisting}
docker info
\end{lstlisting}
	\item Navigate in from the main repository to the \code{bas-observe} submodule's \code{docker} directory
\begin{lstlisting}
cd src/bas-observer/docker
\end{lstlisting}
	\item Start the services in the background using \code{docker-compose}
\begin{lstlisting}
docker-compose up -d
\end{lstlisting}
	\item Check the status of the container
\begin{lstlisting}
docker-compose ps
\end{lstlisting}
	\item Optionally, monitor the log output from \code{InfluxDB}, \code{RabbitMQ}, and \code{Grafana}
\begin{lstlisting}
docker-compose logs -f --tail=200
\end{lstlisting}
	\item Log into \code{Grafana} by navigating to \url{http://localhost:3000/} using a web browser\\
		Username: \code{admin}\\
		Password: \code{testpw}
	\item Import the pre-made dashboards, by navigating to \emph{create} shown as a plus on the left side, and then \emph{Import}
	\item Upload \code{src/bas-observer/docker/grafana-bob-dashboard-avg.json}
	\item Repeat the upload for \code{src/bas-observer/docker/grafana-bob-dashboard-sum.json}
\end{enumerate}

\section{Train the Models}
\label{app:reproduce:train}

\begin{enumerate}
	\item Ensure the services are set up and are running, according to Appendix~\ref{app:reproduce:services}
	\item Activate the virtual environment if necessary, this needs to be repeated for every new terminal session
\begin{lstlisting}
source /tmp/bob-venv/bin/activate
\end{lstlisting}
	\item Start the Collector
\begin{lstlisting}
bob -l INFO --project test collector -a agent0
\end{lstlisting}
	\item Start the the Simulator Agent to import the training dataset
\begin{lstlisting}
bob -l INFO --project test simulate --dump-format new --agent agent0 0 0 data/dataset_training.csv
\end{lstlisting}
	\item To accelerate the import process additional Collectors can be started, without relaying function
\begin{lstlisting}
bob -l INFO --project test collector -a agent0 --no-relay
\end{lstlisting}
	\item Wait until every every window is processed by the Collector\\ (not just until the Simulator Agent terminates)
	\item Train the four different models
		\begin{enumerate}
			\item \gls{lof} Analyser
\begin{lstlisting}
bob -l DEBUG --project test train lof --start "2012-02-27T00:00:00" --end "2012-02-27T01:00:00" -m models/lof_model.json
\end{lstlisting}
			\item \gls{svm} Analyser
\begin{lstlisting}
bob -l DEBUG --project test train svm --start "2012-02-27T00:00:00" --end "2012-02-27T01:00:00" -m models/svm_model.json
\end{lstlisting}
			\item Entropy Analyser
\begin{lstlisting}
bob -l DEBUG --project test train entropy --start "2012-02-27T00:00:00" --end "2012-02-27T01:00:00" -m models/entropy_model.json
\end{lstlisting}
			\item Address Analyser
\begin{lstlisting}
bob -l DEBUG --project test train addr --start "2012-02-27T00:00:00" --end "2012-02-27T01:00:00" -m models/addr_model.json
\end{lstlisting}
		\end{enumerate}
\end{enumerate}

\section{Analyse the Validation and Test Datasets}
\label{app:reproduce:run}

\begin{enumerate}
	\item Ensure the services are set up and are running, according to Appendix~\ref{app:reproduce:services}
	\item Activate the virtual environment if necessary, this needs to be repeated for every new terminal session
\begin{lstlisting}
source /tmp/bob-venv/bin/activate
\end{lstlisting}
	\item Start the Collector
\begin{lstlisting}
bob -l INFO --project test collector -a agent0
\end{lstlisting}
	\item Additional Collector can be started to speed up the processing
\begin{lstlisting}
bob -l INFO --project test collector -a agent0 --no-relay
\end{lstlisting}
	\item Start all Analyser modules
		\begin{enumerate}
			\item \gls{lof} Analyser
\begin{lstlisting}
bob -l INFO --project test analyse lof -m models/lof_model.json
\end{lstlisting}
			\item \gls{svm} Analyser
\begin{lstlisting}
bob -l INFO --project test analyse svm -m models/svm_model.json
\end{lstlisting}
			\item Entropy Analyser
\begin{lstlisting}
bob -l INFO --project test analyse entropy -m models/entropy_model.json
\end{lstlisting}
			\item Address Analyser
\begin{lstlisting}
bob -l INFO --project test analyse addr -m models/addr_model.json
\end{lstlisting}
		\end{enumerate}
	\item Import the validation dataset
\begin{lstlisting}
bob -l INFO --project test simulate --dump-format new --agent agent0 0 0 data/dataset_validation.csv
\end{lstlisting}
	\item Import the test dataset
\begin{lstlisting}
bob -l INFO --project test simulate --dump-format new --agent agent0 0 0 data/dataset_test.csv
\end{lstlisting}
\end{enumerate}

\begin{comment}
\begin{lstlisting}
\end{lstlisting}
\end{comment}

\chapter{Disc with Software and Data}
\label{app:disk}
	\end{appendices}
\end{document}
